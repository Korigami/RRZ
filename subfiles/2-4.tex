%!TEX root = ../RRZ.tex
%
Rozpatrzmy równanie
%
\begin{equation*}
  y^{(n)} + a_{n-1} y^{(n-1)} + \ldots + a_1 y' + a_0 y = 0.
\end{equation*}
%
Możemy je zapisać jako
%
\begin{equation*}
  \begin{bmatrix}
    y' \\
    y'' \\
    \vdots \\
    y^{(n)}
  \end{bmatrix}
  =
  \begin{bmatrix}
    0      & 1      & 0      & \cdots & 0        \\
    0      & 0      & 1      & \cdots & 0        \\
    \vdots & \vdots & \ddots & \ddots & \vdots   \\
    0      & 0      &        & \ddots & 1        \\
    -a_0   & -a_1   & \cdots & \cdots & -a_{n-1}
  \end{bmatrix}
  \cdot
  \begin{bmatrix}
    y \\
    y' \\
    \vdots \\
    y^{(n-1)}
  \end{bmatrix}.
\end{equation*}
%
Wielomian odpowiadający temu równaniu ma postać
%
\begin{equation*}
  \l^n + a_{n-1} \l^{n-1} + \ldots + a_1 \l + a_0 = 0.
\end{equation*}
%
Rozwiązań równania różniczkowego będziemy szukali w zależności od pierwiastków wielomianu. Niech 
$\l_1,\ldots,\l_n$ będą owymi pierwiastkami.
%
\begin{enumerate}
  \item Jeśli $\l_i$ jest rzeczywistym pierwiastkiem jednokrotnym, to $e^{\l_i x}$ jest rozwiązaniem,
  \item Jeśli $\l_i$ jest rzeczywistym pierwiastkiem $p$-krotnym, to $x^i e^{\l_i x}$, są rozwiązaniami dla $i = 0, 
  \ldots, p-1$,
  \item Jeśli $\l_i$ jest zespolonym pierwiastkiem jednokrotnym, to rozwiązaniami są $e^{\real(\l_i)x} \cos 
  \p[\big]{\imag(\l_i) x}$ oraz $e^{\real(\l_i)x} \sin \p[\big]{\imag(\l_i) x}$,
  \item Jeśli $\l_i$ jest zespolonym pierwiastkiem $p$-krotnym, to rozwiązaniami są $x_i e^{\real(\l_i)x} \cos 
  \p[\big]{\imag(\l_i) x}$ oraz $x^i e^{\real(\l_i)x} \sin \p[\big]{\imag(\l_i) x}$ dla $i = 0, \ldots, p-1$.
\end{enumerate}
%
Rozwiązaniami ogólnymi będą kombinacje liniowe powyższych.

Rozważmy teraz odpowiadające równanie niejednorodne
%
\begin{equation*}
  y^{(n)} + a_{n-1} y^{(n-1)} + \ldots + a_1 y' + a_0 y = f(x).
\end{equation*}
%
Rozwiązanie będzie zależało od postaci części niejednorodnej.
%
\begin{enumerate}
  \item Jeśli niejednorodność jest postaci $f(x) = Q_n(x) e^{\l x}$, gdzie $Q_n$ jest wielomianem stopnia $n$, to 
  rozwiązanie szczególne ma postać
  \begin{equation*}
    y_s(x) = x^p P_n e^{\l x},
  \end{equation*}
  przy czym $P_n$ jest wielomianem stopnia co najwyżej $n$, a $p$ jest krotnością $\l$ jako pierwiastka wielomianu.
  \item Jeśli niejednorodność ma postać
  \begin{equation*}
    f(x) = e^{\alpha x} \p[\big]{Q_n(x) \cos(\beta x) + P_m(x) \sin (\beta x)},
  \end{equation*}
  to rozwiązanie szczególne jest postaci
  \begin{equation*}
    y_s(x) = x^p e^{\alpha x} \p[\big]{P_N \cos(\beta x) + Q_N \sin(\beta x)},
  \end{equation*}
  gdzie $N = \max(\deg Q_n, \deg P_m)$, a $p$ jest krotnością $\lambda = \alpha + \beta i$ jako pierwiastka wielomianu.
  \item W przeciwnym przypadku stosujemy metodę uzmienniania stałej. Wtedy
  \begin{equation*}
    \begin{bmatrix}
      y_1         & y_2         & \cdots & y_n         \\
      y_1'        & y_2'        & \cdots & y_n'        \\
      \vdots      & \vdots      & \ddots & \vdots      \\
      y_1^{(n-1)} & y_2^{(n-1)} & \cdots & y_n^{(n-1)}
    \end{bmatrix}
    \cdot
    \begin{bmatrix}
      c_1'(x) \\
      c_2'(x) \\
      \vdots  \\
      c_n'(x)
    \end{bmatrix}
    =
    \begin{bmatrix}
    0 \\
    0 \\
    \vdots \\
    f(x)
    \end{bmatrix}.
  \end{equation*}
  Wówczas
  \begin{equation*}
    y_s = c_1 y_1 + \ldots + c_n y_n.
  \end{equation*}
\end{enumerate}
%
\begin{theorem}[Zasada Duhamela]
  Jeśli $y' = A(t)y + B(t)$, to rozwiązania są dane wzorem
%
  \begin{equation*}
    y_0(t) = \int_{t_0}^t M(t,u) B(u) du,
  \end{equation*}
%
  gdzie $M(t,u)$ jest macierzą rozwiązującą równania $y' = A(t)y$ taką, że $M(t,t) =~E$.
\end{theorem}
































