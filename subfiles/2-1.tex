%!TEX root = ../RRZ.tex

\begin{theorem}[Peano]
  Niech $y' = f(y,t)$, gdzie $y(t_0) = y_0$ oraz
  \[
    f \colon H = \overline{B} (y_0, b) \times [t_0 - a, t_0 + a] \longrightarrow \R^m.
  \]
  Załóżmy, że funkcja $f$ jest ciągła i oznaczmy
  \[
    M = \sup \big\{ \norm[\big]{f(y,t)} : (y, t) \in H \big\}.
  \]
  Wówczas dla $\alpha = \min(a, b/M)$ istnieje rozwiązanie $y(t)$ określone na
  przedziale $[t_0 - \alpha, t_0 + \alpha]$, spełniające warunek początkowy $y(t_0) = y_0$.
\end{theorem}

\begin{theorem}[Picard-Lindelöf]
  Niech $y' = f(y,t)$, $y(t_0) = y_0$, gdzie
  \[
    f \colon H = \overline{B} (y_0, b) \times [t_0 - a, t_0 + a] \longrightarrow \R^m.
  \]
  Załóżmy, że funkcja $f$ jest ciągła oraz lipszycowska ze względu na $y$, to znaczy
  \[
    \exists L \ \forall (y_1,t), (y_2,t) \in H \quad
    \norm[\big]{f(y_1,t) - f(y_2,t)} \leq L \cdot \norm{y_1 - y_2}.
  \]
  Oznaczmy ponadto
  \[
    M = \sup \big\{ \norm[\big]{f(y,t)} : (y, t) \in H \big\}.
  \]
  Wówczas dla dowolnego $\alpha < \min(a, b/M, 1/L)$ istnieje dokładnie jedno rozwiązanie
  zagadnienia Cauchy'ego z warunkiem początkowym $y(t_0) = y_0$ określone na przedziale
  $[t_0 - \alpha, t_0 + \alpha]$.
\end{theorem}

\begin{lemma}[o zgodności rozwiązań]
  Niech $y' = f(y,t)$, gdzie funkcja $f \colon U \to \R^m$ jest ciągła i lokalnie lipszycowska
  względem $y$. Niech $(y_0,t_0) \in U$. Jeśli $y_1(t), y_2(t)$ są rozwiązaniami określonymi
  odpowiednio na $I_1, I_2$, spełniającymi ten sam warunek początkowy $y_1(t_0) = y_2(t_0) = y_0$,
  to $y_1 \equiv y_2$ na $I_1 \cap I_2$.
\end{lemma}

\begin{lemma}[o przedłużaniu przez koniec]
  Niech $y' = f(y,t)$, gdzie funkcja $f \colon \R^{m+1} \supset U \to \R^m$ jest ciągła i lokalnie
  lipszycowska względem $y$. Niech $y(t)$ -- rozwiązanie, $T$ -- koniec $\domain y$, granica
  $\lim_{t \to T} y(t) = y_T$ istnieje oraz $(y_T, T) \in U$. Wówczas $y$ rozszerza się na przedział
  zawierający $T$ we wnętrzu.
\end{lemma}

\begin{theorem}[o przedłużaniu przez koniec] \label{T: TOPK}
  Niech $y' = f(y,t)$, gdzie $f \colon \R^{m+1} \supset U \to \R^m$ jest ciągła i lokalnie
  lipszycowska względem $y$. Załóżmy, że rozwiązanie $y(t)$ jest określone na pewnym przedziale,
  którego końcem jest $T \in \R$. Załóżmy dalej, że istnieje zbiór zwarty $K \subset U$ oraz
  $\varepsilon > 0$, taki, że
  \[
    \forall t \in \domain y \cap [T - \varepsilon, T + \varepsilon] \quad \p[\big]{y(t), t} \in K.
  \]
  Wtedy $y$ rozszerza się na przedział zawierający $T$ we wnętrzu.
\end{theorem}

\begin{theorem}[o rozwiązaniu wysyconym]
  Niech $y' = f(y,t)$, gdzie funkcja $f \colon \R^{m+1} \supset U \to \R^m$ jest ciągła i lokalnie
  lipszycowska względem $y$ oraz $(y_0, t_0) \in U$. Wówczas istnieje rozwiązanie $y_{\mathrm{max}}$ zwane
  \emph{wysyconym}, określone na przedziale otwartym, spełniające warunek początkowy $y_{\mathrm{max}}(t_0)
  = y_0$ i takie, że jeśli $y$ jest dowolnym rozwiązaniem spełniającym warunek $y(t_0) = y_0$, to
  $\domain y \subset \domain y_{\mathrm{max}}$ oraz $y$ jest obcięciem $y_{\mathrm{max}}$.
\end{theorem}