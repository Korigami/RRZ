%!TEX root = ../RRZ.tex
%
\begin{definition}
  \emph{Widmem} macierzy $A$ nazywamy zbiór jej wartości własnych wraz z krotnościami i oznaczamy $\spectrum (A)$.
\end{definition}
%
\begin{theorem}[Hamilton-Cayley] \label{T: H-C}
  Dla każdej macierzy $A$ zachodzi $\mychi_A(A) = 0$, gdzie $\mychi_A$ jest wielomianem charakterystycznym macierzy $A$.
\end{theorem}
%
\begin{theorem}[Spektralne dla wielomianów]
  Niech $A$ będzie macierzą o wartościach własnych $\lambda_1,\ldots,\lambda_n$ z krotnościami $q_1,\ldots,q_n$. Wtedy 
  istnieją macierze $M_{k,l}$ dla $1 \leq k \leq n$, $0 \leq l \leq q_k-1$, zwane \emph{spektralnymi} takie, że dla 
  każdego wielomianu $f$ zachodzi:
%
  \begin{equation*}
    f(A) = \sum_{k=1}^{n} \sum_{l=0}^{q_k-1} M_{k,l} \cdot f^{(l)}(\lambda_k).
  \end{equation*}
\end{theorem}
%
\begin{proof}
  W celu udowodnienia twierdzenia będzie potrzebny lemat pomocniczy.
%
  \begin{nestedlemma} \label{L: 1.4.2}
    Macierze $M_{k,l}$ są jednoznacznie wyznaczone przez tezę twierdzenia spektralnego dla
    wielomianów postaci $f(z) = z^r$, gdzie $r = 0, \ldots, m-1$, a~$m$ jest stopniem wielomianu $f$.
  \end{nestedlemma}
%
  \begin{nestedproof}
    Otrzymujemy układ równań z niewiadomymi $M_{k,l}$, czyli
%   
    \begin{equation*}
      A^r = \sum_{k=1}^n \sum_{l=0}^{q_k-1} M_{k,l} \cdot r (r-1) \cdots (r-l+1) \cdot \lambda_k^{r-l}.
    \end{equation*}
%   
    Teza lematu oznacza, że układ ten jest oznaczony. Pokażemy liniową niezależność wierszy. Wybierzmy współczynniki 
    $c_r$ dla $r = 0, \ldots, m-1$, tak aby kombinacja liniowa wierszy z tymi współczynnikami wynosiła $0$, czyli dla
    każdych $k = 1,\ldots,n$ oraz $l = 0, \ldots, q_k-1$ jest
%   
    \begin{equation*}
      \sum_{r=0}^{m-1} c_r \cdot r (r-1) \cdots (r-l+1) \lambda_k^{r-l} = 0.
    \end{equation*}
%
    Rozważmy teraz wielomian
%    
    \begin{equation*}
      w(z) = \sum_{r=0}^{m-1} c_r z^r.
    \end{equation*}
%
    Otrzymaliśmy, że $w^{(l)}(\lambda_k) = 0$, czyli 
    $\lambda_k$ jest zerem z krotnością co najmniej $q_k$, a zatem suma krotności zer wielomianu $w$ jest równa co 
    najmniej $\sum_{k=1}^n q_k = m$, co jest sprzecznością, bo stopień wielomianu był co najwyżej $m-1$.
  \end{nestedproof}
%
  Twierdzenie zostanie udowodnione indukcyjnie ze względu na stopień $f$.
  
  Przypuśćmy, że twierdzenie zachodzi dla wielomianów stopnia mniejszego od $m+r$, gdzie $r \geq 0$. Z lematu 
  \ref{L: 1.4.2} teza zachodzi dla $r=0$. Zwróćmy uwagę, że obie strony twierdzenia są liniowe względem $f$. 
  Wystarczy więc pokazać je dla układu rozpinającego przestrzeń wielomianów stopnia mniejszego niż $m+r$. W~celu 
  pokazania, że twierdzenie zachodzi również dla wielomianów stopnia $m+r$, wystarczy pokazać dla $f_r(z) = z^r 
  \mychi_A(z)$, bo każdy wielomian
%
  \begin{equation*}
    f(z) = a_{m+r} z^{m+r} + \ldots + a_1 z + a_0
  \end{equation*}
%
  można zapisać jako
%
  \begin{equation*}
    f(z) = a_{m+r} f_r(z) + P(z),
  \end{equation*}
%
  gdzie $P$ jest wielomianem stopnia mniejszego niż $m+r$. Zauważmy, że
%
  \begin{equation*}
    L = f_r(A) = A^r \cdot \mychi_A(A) \overset{\ref{T: H-C}}{=} 0, \qquad 
    P = \sum_{k=1}^n \sum_{l=0}^{q_k-1} M_{k,l} \cdot f^{(l)}(\lambda_k),
  \end{equation*}
%
  a ponadto $f_r(z) = (z-\lambda_k)^{q_k} \cdot Q(z)$. Pochodne rzędu niższego od $q_k$ składają się z sum członów, w 
  których $(z-\lambda_k)$ występuje w dowolnej potędze, więc zerują się przy podstawieniu $z=\lambda_k$.
%
\end{proof}
%
\begin{theorem}[Spektralne dla funkcji analitycznych]
  Niech
%
  \begin{equation*}
    f(z) = \sum_{n=0}^{\infty} a_n z^n, \quad \abs{z} < R.
  \end{equation*}
%
  Załóżmy, że $\spectrum(A) \subset D(0,R)$. Wówczas szereg $f(A)$ zbiega i zachodzi teza twierdzenia spektralnego dla 
  wielomianów:
%
  \begin{equation*}
    f(A) = \sum_{k=1}^{n} \sum_{l=0}^{q_k-1} M_{k,l} \cdot f^{(l)}(\lambda_j).
  \end{equation*}
\end{theorem}
%
\begin{proof}
  Oznaczmy
%
  \begin{equation*}
    f_N(z) = \sum_{n=0}^N a_n z^n.
  \end{equation*}
%
  Korzystając z twierdzenia spektralnego dla wielomianów, dostajemy
%
  \begin{multline*}
    f(A) = \lim_{N\to\infty} f_N(A) 
    = \lim_{N\to\infty} \sum_{k=1}^{n} \sum_{l=0}^{q_k-1} M_{k,l} \cdot f_N^{(l)}(\lambda_j) = \\
    = \sum_{k=1}^{n} \sum_{l=0}^{q_k-1} M_{k,l} \cdot \lim_{N\to\infty} f_N^{(l)}(\lambda_j)
    = \sum_{k=1}^{n} \sum_{l=0}^{q_k-1} M_{k,l} \cdot f^{(l)}(\lambda_j). \tag*{\qedhere}
  \end{multline*}
\end{proof}
































