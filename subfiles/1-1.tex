%!TEX root = ../RRZ.tex

\begin{theorem}[O ciągłej zależności od parametru]
  Niech
  \[
    y' = f(y,t,\lambda), \qquad f \colon \R^{m+1} \times \R^l \supset U \times B_l(\lambda_0,c) \longrightarrow \R^m,
  \]
  gdzie $f$ jest funkcją ciągłą oraz $c>0$. Niech $y(t,\lambda_0)$ będzie rozwiązaniem równania $y' = f(y,t,\lambda_0)$ 
  z warunkiem początkowym $(y_0,t_0)$ określonym na \emph{zwartym} przedziale $I$ zawierającym $t_0$. Wybierzmy $b > 0$ 
  i rozważmy zbiór
  \[
    R_b = \big\{ (y,t) : t \in I \text{ oraz } \norm[\big]{ y - y(t,\lambda_0) } < b \big\}.
  \]
  Załóżmy dalej, że
  \begin{enumerate}
    \item istnieje $L \geq 0$, że dla wszystkich $(y_1,t),(y_2,t) \in R_b$ jest:
    \[
      \norm[\big]{ f(y_1,t,\lambda_0) - f(y_2,t,\lambda_0) } \leq L \cdot \norm{y_1-y_2},
    \]
    \item dla dowolnego $\varepsilon>0$ istnieje $\delta>0$, że dla $(y,t) \in R_b$ zachodzi:
    \[
      \forall \lambda \quad \norm{\lambda - \lambda_0} < \delta \implies 
      \norm[\big]{ f(y,t,\lambda) - f(y,t,\lambda_0) } < \varepsilon.
    \]
  \end{enumerate}
  Wówczas istnieje stała $c^{\ast} > 0$ taka, że
  \begin{enumerate}
    \item jeśli $\norm{\lambda-\lambda_0} < c^{\ast}$, to $y(t,\lambda)$ jest określone na $I$,
    \item jeśli $\lambda_n \to \lambda_0$, to $y(t,\lambda_n) \rightrightarrows y(t,\lambda_0)$ na $I$.
  \end{enumerate}
\end{theorem}
\begin{proof}
  W dowodzie wykorzystamy poniższy lemat.
  \begin{lemma} \label{L: Kluczowy}
    Przy założeniach twierdzenia przypuśćmy, że
    \begin{enumerate}
      \item na pewnym przedziale $J \subset I$ jeśli $t \in J$, to $\p[\big]{y(t,\lambda),t} \in R_b$,
      \item dla każdego $\varepsilon > 0$ oraz $(y,t) \in R_b$ zachodzi:
      $ \norm[\big]{ f(y,t,\lambda) - f(y,t,\lambda_0) } \leq \varepsilon $.
    \end{enumerate}
    Wtedy dla każdego $t \in J$ jest:
    \[
      \norm[\big]{ y(t,\lambda) - y(t,\lambda_0) } \leq \varepsilon \cdot e^{L \abs{t-t_0}} \cdot \abs{t-t_0}.
    \]
  \end{lemma}
  \begin{nestedproof}
    \begin{align*}
      \norm[\big]{ y(t,\lambda) - y(t,\lambda_0) }
      &= \norm*{ \int_{t_0}^t f \p[\big]{ y(u, \lambda), u, \lambda } du - 
          \int_{t_0}^t f\p[\big]{ y(u,\lambda_0), u, \lambda_0 } du} \\
      &\leq \abs*{ \int_{t_0}^t \norm[\Big]{ f \p[\big]{ y(u,\lambda), u, \lambda } - 
          f\p[\big]{ y(u,\lambda_0), u, \lambda_0 } } du } \\
      &\leq \abs*{ \int_{t_0}^t \norm[\Big]{ f \p[\big]{ y(u,\lambda), u, \lambda } - 
          f\p[\big]{ y(u,\lambda), u, \lambda_0 } } du } \\
      &\qquad+ \abs*{ \int_{t_0}^t \norm[\Big]{ f \p[\big]{ y(u,\lambda), u, \lambda_0 } - 
          f\p[\big]{ y(u,\lambda_0), u, \lambda_0 } } du } \\
      &\leq \abs*{\int_{t_0}^t \varepsilon \, du} + 
          \abs*{ \int_{t_0}^t L \cdot \norm[\big]{ y(u,\lambda) - y(u,\lambda_0) } du } \\
      &\leq \underbrace{\varepsilon \cdot \abs*{t-t_0}}_K + 
          \abs*{ \int_{t_0}^t L \cdot \norm[\big]{ y(u,\lambda) - y(u,\lambda_0) } du }.
    \end{align*}
    Z nierówności Gronwalla ze stałą $K$ wynika teza.
  \end{nestedproof}
  Wybierzmy $\varepsilon > 0$ taki, że $\varepsilon \cdot \abs{I} \cdot e^{L \abs{I}} \leq \frac b2$ oraz oznaczmy 
  $c^{\ast} = \min(c,\delta_{\varepsilon})$. Weźmy $\lambda > 0$ taką, że $\norm{\lambda - \lambda_0} < 
  \delta_{\varepsilon}$. Niech $J$ będzie maksymalnym podprzedziałem $I$, na którym zachodzi:
  \[
    \forall t \in J \quad \p[\big]{ y(t,\lambda), t } \in R_b.
  \]
  Wtedy z lematu \ref{L: Kluczowy} dostajemy:
  \[
    \forall t \in J \quad \norm[\big]{ y(t,\lambda) - y(t,\lambda_0) } \leq \frac b2.
  \]
  Przypuśćmy, że jeden z końców $J$, nazwijmy go $\alpha$, należy do wnętrza $I$. Wówczas z twierdzenia \ref{T: TOPK}, 
  $y(t,\lambda)$ przedłuża się na przedział $\widetilde{J}$ zawierający $\alpha$ we wnętrzu. Co więcej, ponieważ 
  $\norm[\big]{ y(t,\lambda) - y(t,\lambda_0) }$ jest funkcją ciągłą względem $t$, to pozostaje mniejsza od $b$ na 
  pewnym otoczeniu $\alpha$. Zatem $J$ nie był przedziałem maksymalnym, chyba że $J = I$. Pokazaliśmy więc pierwszą 
  część tezy.
  
  Jeśli $\lambda_n \to \lambda_0$, to z założenia drugiego istnieje ciąg $\varepsilon_n \to 0$ taki, że
  \[
    \forall (y,t) \in R_b \ \forall n \quad \norm[\big]{ f(y,t,\lambda) - f(y,t,\lambda_0) } < \varepsilon_n.
  \]
  Z lematu \ref{L: Kluczowy} dostajemy:
  \[
    \forall t \in I \quad \norm[\big]{ y(t,\lambda) - y(t,\lambda_0) } \leq
    \varepsilon_n \cdot \abs{I} \cdot e^{L \abs{I}} \xrightarrow{n\to\infty} 0,
  \]
  co dowodzi drugiej części tezy.
\end{proof}


































