%!TEX root = ../RRZ.tex

\documentclass[titleauthor]{mwrep}

\usepackage[utf8]{inputenc}

\usepackage[T1, nomathsymbols]{polski}

\usepackage{lmodern}

\usepackage{amsmath, amsthm, amssymb, amsfonts}
  \numberwithin{equation}{section}

\usepackage{mathtools}
  \DeclarePairedDelimiter\abs{\lvert}{\rvert}
  \DeclarePairedDelimiter\norm{\lVert}{\rVert}
  \DeclarePairedDelimiter\p{(}{)}
  \DeclareMathOperator{\real}{Re}
  \DeclareMathOperator{\imag}{Im}
  \DeclareMathOperator{\domain}{Dm}
  \DeclareMathOperator{\spectrum}{sp}

\usepackage{thmtools}
  \declaretheoremstyle[
    spaceabove = \topsep,
    spacebelow = \topsep,
    headfont   = \bfseries,
    notefont   = \bfseries,
    notebraces = {(}{)},
    bodyfont   = \normalfont,
  ]{normal}
  
  \declaretheoremstyle[
    spaceabove = \topsep,
    spacebelow = \topsep,
    headfont   = \scshape,
    notefont   = \scshape,
    notebraces = {(}{)},
    bodyfont   = \normalfont,
  ]{nested}
  
  \let\proof\relax
  
  \declaretheorem[style = normal, name = Twierdzenie,   parent  = section]{theorem}
  \declaretheorem[style = normal, name = Stwierdzenie, sibling  = theorem]{statement}
  \declaretheorem[style = normal, name = Lemat,        sibling  = theorem]{lemma}
  \declaretheorem[style = normal, name = Definicja,    sibling  = theorem]{definition}
  \declaretheorem[style = normal, name = Wniosek,      sibling  = theorem]{conclusion}
  \declaretheorem[style = normal, name = Przykład,     sibling  = theorem]{example}
  \declaretheorem[style = normal, name = Dowód,        numbered = no, qed = \ensuremath{\blacksquare}]{proof}
  
  \declaretheorem[style = nested, name = Lemat,  sibling = theorem]{nestedlemma}
  \declaretheorem[style = nested, name = Dowód, numbered = no, qed = \ensuremath{\square}]{nestedproof}

\usepackage{parskip}

\usepackage{enumitem}
  \setlist{leftmargin=0.75cm}

\newcommand*{\R}{\mathbb{R}}
\newcommand*{\N}{\mathbb{N}}
\newcommand*{\Q}{\mathbb{Q}}
\newcommand*{\Z}{\mathbb{Z}}
\newcommand*{\C}{\mathbb{C}}
\newcommand*{\mychi}{\protect\raisebox{2pt}{$\chi$}}

\pagestyle{uheadings}
\makeatletter
\renewcommand\heading@font\scshape
\makeatother

\title{
  \huge \textbf{Równania różniczkowe zwyczajne} \\
  \Large Opracowanie zagadnień na egzamin
}
\author{ KJG }
\date{ Wersja z \today }
































