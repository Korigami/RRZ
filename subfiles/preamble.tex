%!TEX root = ../RRZ.tex

\documentclass{mwrep}

\usepackage[utf8]{inputenc}

\usepackage[T1, nomathsymbols]{polski}

\usepackage{lmodern}

\usepackage{amsmath, amsthm, amssymb, amsfonts}
    \numberwithin{equation}{section}

\usepackage{mathtools}
    \DeclarePairedDelimiter\abs{\lvert}{\rvert}
    \DeclarePairedDelimiter\norm{\lVert}{\rVert}
    \DeclarePairedDelimiter\p{(}{)}
    \DeclareMathOperator{\Domain}{Dm}
    \DeclareMathOperator{\spectre}{sp}

\usepackage{thmtools}
    \declaretheoremstyle[
      spaceabove = \topsep,
      spacebelow = \topsep,
      headfont   = \bfseries,
      notefont   = \bfseries,
      notebraces = {(}{)},
      bodyfont   = \normalfont,
    ]{mystyle}
    
    \let\proof\relax
    
    \declaretheorem[style = mystyle, name = Twierdzenie,   parent  = section]{theorem}
    \declaretheorem[style = mystyle, name = Stwierdzenie, sibling  = theorem]{statement}
    \declaretheorem[style = mystyle, name = Lemat,        sibling  = theorem]{lemma}
    \declaretheorem[style = mystyle, name = Definicja,    sibling  = theorem]{definition}
    \declaretheorem[style = mystyle, name = Wniosek,      sibling  = theorem]{conclusion}
    \declaretheorem[style = mystyle, name = Przykład,     sibling  = theorem]{example}
    \declaretheorem[style = mystyle, name = Dowód,        numbered = no, qed = \ensuremath{\blacksquare}]{proof}
    \declaretheorem[style = mystyle, name = Dowód,        numbered = no, qed = \ensuremath{\square}]{nestedproof}

\usepackage{parskip}

\newcommand*{\R}{\mathbb{R}}
\newcommand*{\N}{\mathbb{N}}
\newcommand*{\Q}{\mathbb{Q}}
\newcommand*{\Z}{\mathbb{Z}}
\newcommand*{\C}{\mathbb{C}}
\newcommand*{\mychi}{\protect\raisebox{2pt}{$\chi$}}

\pagestyle{uheadings}
\makeatletter
\renewcommand\heading@font\scshape
\makeatother

\title{
  \huge \textbf{Równania różniczkowe zwyczajne} \\
  \Large Opracowanie zagadnień na egzamin
}
\author{ KJG }
\date{ Wersja z \today }


































