%!TEX root = ../RRZ.tex

\begin{theorem}[O różniczkowalnej zależności od parametru]
  Niech
  %
  \begin{equation*}
    y' = f(y,t,\lambda), \qquad
    f \colon \R^{m+1} \times \R \supset U \times (\lambda_0 - c, \lambda_0 + c) \longrightarrow \R^m,
  \end{equation*}
  %
  gdzie $f$ jest funkcją ciągłą względem $y,t,\lambda$ oraz klasy $C^1$ względem $y,\lambda$. Ustalmy warunek 
  początkowy $(y_0,t_0)$ i oznaczmy przez $y(t,\lambda)$ rozwiązanie równania
  %
  \begin{equation*}
    \frac{\partial y(t,\lambda)}{\partial t} = f(y,t,\lambda)
  \end{equation*}
  %
  z warunkiem początkowym $y(t_0,\lambda) = y_0$, określone na ustalonym i \emph{zwartym} przedziale $I$. Wówczas na 
  przedziale $I$ istnieje ciągła funkcja
  %
  \begin{equation*}
    z(t,\lambda_0) = \frac{\partial y(t,\lambda)}{\partial \lambda} \Big\vert_{\lambda = \lambda_0}
  \end{equation*}
  %
  oraz zachodzi równość
  %
  \begin{equation*}
    \frac{\partial z(t,\lambda_0)}{\partial t} =
    \frac{\partial^2 y(t,\lambda)}{\partial t \partial \lambda} \Big\vert_{\lambda = \lambda_0} =
    \frac{\partial^2 y(t,\lambda)}{\partial \lambda \partial t} \Big\vert_{\lambda = \lambda_0}.
  \end{equation*}
  %
\end{theorem}

\begin{proof}
  Łatwy dowód czytelnik sporządzi samodzielnie jako ćwiczenie.
\end{proof}
































