%!TEX root = ../RRZ.tex
%
\begin{theorem}[O różniczkowalnej zależności od parametru]
Niech
%
  \begin{equation*}
    y' = f(y,t,\lambda), \qquad
    f \colon \R^{m+1} \times \R \supset U \times (\lambda_0 - c, \lambda_0 + c) \longrightarrow \R^m,
  \end{equation*}
%
  gdzie $f$ jest funkcją ciągłą względem $y,t,\lambda$ oraz klasy $C^1$ względem $y,\lambda$. Ustalmy warunek 
  początkowy $(y_0,t_0)$ i oznaczmy przez $y(t,\lambda)$ rozwiązanie równania
%
  \begin{equation*}
    \frac{\partial y(t,\lambda)}{\partial t} = f(y,t,\lambda)
  \end{equation*}
%
  z warunkiem początkowym $y(t_0,\lambda) = y_0$, określone na ustalonym i \emph{zwartym} przedziale $I$. Wówczas na 
  przedziale $I$ istnieje ciągła funkcja
%
  \begin{equation*}
    z(t,\lambda_0) = \frac{\partial y(t,\lambda)}{\partial \lambda} \bigg\vert_{\lambda = \lambda_0}
  \end{equation*}
%
oraz zachodzi równość
%
  \begin{equation*}
    \frac{\partial z(t,\lambda_0)}{\partial t} =
    \frac{\partial^2 y(t,\lambda)}{\partial t \partial \lambda} \bigg\vert_{\lambda = \lambda_0} =
    \frac{\partial^2 y(t,\lambda)}{\partial \lambda \partial t} \bigg\vert_{\lambda = \lambda_0}.
  \end{equation*}
\end{theorem}

\begin{proof}
  Niech $y_\l(t) \in R_b$. Oznaczmy
%
  \begin{equation*}
    w_\l(t) \coloneqq \frac{ y_\l(t) - y_{\l_0}(t) }{ \l - \l_0 }
  \end{equation*}
%
  dla $\l \ne \l_0$. Wtedy
%
  \begin{multline*}
    \pd{t} w_\l(t) = \frac{ \pd{t} y_\l(t) - \pd{t} y_{\l_0}(t) }{ \l - \l_0 } = \frac{ f_\l\p[\big]{y_\l(t), t} - 
    f_{\l_0} \p[\big]{y_{\l_0}(t), t} }{ \l - \l_0 } = \\
    =  \frac{ f_\l \p[\big]{y_{\l_0}(t) + (\l - \l_0)w_\l(t), t} - 
    f_{\l_0} \p[\big]{y_{\l_0}(t), t} }{ \l - \l_0 } .
  \end{multline*}
%
  Rozważmy dalej funkcję
%
  \begin{equation*}
    F_\l(w, t) \coloneqq \frac{ f_\l \p[\big]{y_{\l_0}(t) + (\l - \l_0)w, t} - 
      f_{\l_0}\p[\big]{y_{\l_0}(t), t} }{ \l - \l_0 }.
  \end{equation*}
%
  Zwróćmy uwagę, że $w$ w definicji $F$ jest symbolem argumentu, a nie funkcji.

  Pokażemy, że istnieje taka stała $c^* > 0$, że o ile $ \abs{ \l - \l_0} \le c^*$, to dziedzina funkcji $F$ dla 
  każdego $\l$ spełniającego ten warunek jest zbiorem zwartym, a ponadto $w_\l(t)$ zawiera się w tej dziedzinie dla 
  każdego $t \in I$. 

  Z lematu \ref{L: Kluczowy} wiemy, że istnieją stałe $c_1, K_1 > 0$ takie, że
%
  \begin{equation*}
    \abs{ \l - \l_0 } < c_1 \implies \norm[\big]{ w_\l(t) } \le K_1.
  \end{equation*}
%
  Z kolei jeśli $\abs{ \l - \l_0 } \le \frac{ b }{ 2K_1 }$, to dla $w$ takich, że $\norm{ w } \le 2K_1$ zachodzi
%
  \begin{equation*}
    \norm[\big]{ y_{\l_0}(t) + (\l - \l_0)w - y_{\l_0}(t) } = \norm[\big]{ (\l - \l_0)w  } = \abs{ \l - \l_0 } \cdot 
    \norm{ w } \le \frac{ b }{ 2K_1 } \cdot 2K_1 = b.
  \end{equation*}
%
  Oznacza to, że dla każdego $t \in J$ jest
%
  \begin{equation*}
    \p[\big]{ y_{\l_0}(t) + (\l - \l_0)w, t} \in R_b.
  \end{equation*} 
%
  Niech $c^* = \min\p[\big]{c_1, \frac{ b }{ 2K_1 } }$. Wtedy $F$ jest zdefiniowana na zbiorze
%
  \begin{equation*}
    \overline{B}(0, 2K_1) \times I \times \p[\big]{ (\l_0 - c^*, \l_0 + c^*) \setminus \{ \l_0 \} } \text{.}
  \end{equation*}
%
Zdefiniujmy $F$ dla $\l = \l_0$ tak, by była ciągła względem $\l$ w tym punkcie. Wtedy
%
  \begin{align*}
    \lim_{\l \to \l_0} F_\l(w, t) &= \lim_{\l \to \l_0} \frac{ f_\l\p[\big]{y_{\l_0}(t) + (\l - \l_0)w, t} -  
    f_{\l_0} \p[\big]{y_{\l_0}(t), t} }{ \l - \l_0 } \\
    &=  \lim_{\l \to \l_0} \frac{ f_\l \p[\big]{y_{\l_0}(t) + (\l - \l_0)w, t} -  f_\l \p[\big]{y_{\l_0}(t), t} }{ w(\l 
    - \l_0) } \cdot w \, + \\
    &\qquad+ \lim_{\l \to \l_0} \frac{ f_\l\p[\big]{y_{\l_0}(t), t} - f_{\l_0} \p[\big]{y_{\l_0}(t)} }{ \l - \l_0 } \\
    &= \pd{ y } f_{\l_0} (y, t) \big\vert_{ y = y_{\l_0}(t) } \cdot w + \pd{ \l } f_{\l}( y_{\l_0}(t), t) 
    \big\vert_{ \l = \l_0 },
  \end{align*}
%
  przy czym powyższe pochodne istnieją, gdyż z założenia funkcja $f$ jest klasy $C^1$ względem zmiennych $y$ oraz $\l$. 
  Niech
%
  \begin{equation*}
    F_{\l_0}(w, t) \coloneqq \lim_{\l \to \l_0} F_\l(w, t).
  \end{equation*}
%
  Wtedy $F$ jest zdefiniowana na zbiorze
%
  \begin{equation*}
    \overline{B}(0, 2K_1) \times I \times (\l_0 - c^*, \l_0 + c^*),
  \end{equation*}
%
  który jest zwarty dla każdego ustalonego $\l$.
%
  \begin{nestedlemma}
    Funkcja $F$ zdefiniowana powyżej spełnia założenia twierdzenia o~ciągłej zależności od parametru przy równaniu 
    różniczkowym $\pd{t} w = F_\l(w, t)$.
  \end{nestedlemma}
%
  \begin{nestedproof}
    Nadobowiązkowy dowód Czytelnik wykona samodzielnie.
  \end{nestedproof}
%
  Z powyższych rozważań oraz na mocy Twierdzenia Picarda-Lindelöfa, dla każdego $\l \in  (\l_0 - c^*, \l_0 + c^*)$ 
  istnieje rozwiązanie powyższego równania różniczkowego, a także istnieje jednoznaczne rozwiązanie dla $\l = \l_0$ 
  określone w pewnym przedziale $J \subset I$ (dla ustalenia uwagi, niech $J$ będzie maksymalnym możliwym). Co~więcej, 
  z twierdzenia o ciągłej zależności mamy, że dla każdego ciągu $\l_n \to \l_0$ zachodzi $w_{\l_n}(t) \rightrightarrows 
  w_{\l_0}(t)$. Stąd
%
  \begin{equation*}
    w_{\l_0}(t) = \lim_{n \to \infty} w_{\l_n}(t) = \lim_{\l \to \l_0} w_{\l}(t) = \lim_{\l \to \l_0} \frac{ y_\l(t) - 
    y_{\l_0}(t) }{ \l - \l_0 } = \pd{ \l } y_\l(t) \big\vert_{ \l = \l_0 }.
  \end{equation*}
%
  Zatem $w_{\l_0}$ jest szukaną funkcją $z$ z tezy twierdzenia.

Równość w twierdzeniu zachodzi, gdyż
%
  \begin{align*}
    \pd{t} w_{\l_0}(t) &= \pd{t} \lim_{\l \to \l_0}  \frac{ y_\l(t) - y_{\l_0}(t) }{ \l - \l_0 } \\ &= \pd{t} \lim_{n 
    \to 
    \infty}
    \frac{ y_{\l_n}(t) - y_{\l_0}(t) }{ \l_n - \l_0 } \\
    &= \lim_{n \to \infty} \frac{  \pd{t} y_{\l_n}(t) -  \pd{t} y_{\l_0}(t) }{ \l_n - \l_0 } \\ &= \lim_{\l \to \l_0} 
    \frac{  
    \pd{t} y_{\l_n}(t) -  \pd{t} y_{\l_0}(t) }{ \l_n - \l_0 }
    = \pd{\l } \pd{t} y_\l (t) \big\vert_{ \l = \l_0 }.
  \end{align*}
%
  Czyli istotnie $\pd{t} w_{\l_0}(t) = \pd{\l } \pd{t} y_\l (t) \big\vert_{ \l = \l_0 }$.
\end{proof}


























