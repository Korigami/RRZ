%!TEX root = ../RRZ.tex
%
\begin{definition}
  \emph{Równaniem liniowym} nazywamy równanie postaci
%
  \begin{equation*}
    y' = A(t)y + B(t),
  \end{equation*}
%
  gdzie $A(t)$ jest macierzą $m \times m$, $B(t)$ wektorem z $\R^m$, a ich współczynniki są funkcjami ciągłymi 
  określonymi na przedziale otwartym $I \subset \R$.
\end{definition}
%
\begin{lemma}
  Dla dowolnego warunku początkowego $(y_0,t_0)$, gdzie $y_0 \in \R^m$ oraz $t_0 \in I$, dziedziną rozwiązania 
  wysyconego jest $I$.
\end{lemma}
%
\begin{definition}
  Równanie $y' = A(t)y$ nazywamy \emph{jednorodnym}, a równanie $y' = A(t)y + B(t)$ (odpowiadającym) 
  \emph{niejednorodnym}.
\end{definition}
%
\begin{theorem}
  Zbiór rozwiązań równania jednorodnego jest podprzestrzenią liniową $C^0(I,\R^m)$, a zbiór rozwiązań równania 
  niejednorodnego jej warstwą.
\end{theorem}
%
Niech $V$ oznacza zbiór rozwiązań wysyconych równania $y' = A(t)y$.
%
\begin{statement}
  Niech $\{y_1, \ldots, y_n\} \subset V$. Poniższe warunki są równoważne:
%
  \begin{enumerate}
    \item Zbiór $\{y_1, \ldots, y_n\}$ jest liniowo niezależny.
    \item Dla dowolnego $t \in I$ zbiór $\big\{y_1(t), \ldots, y_n(t)\big\}$ jest liniowo niezależny w $\R^m$.
    \item Istnieje $t \in I$, że zbiór $\big\{y_1(t), \ldots, y_n(t)\big\}$ jest liniowo niezależny w $\R^m$.
  \end{enumerate}
\end{statement}
%
\begin{statement}
  Niech $\{y_1, \ldots, y_k\} \subset V$. Wtedy
%
  \begin{enumerate}
    \item Jeśli $\{y_1, \ldots, y_k\}$ jest liniowo niezależny, to $k \leq m$.
    \item Jeśli $\{y_1, \ldots, y_k\}$ rozpina $V$, to $k \geq m$.
  \end{enumerate}
\end{statement}
%
\begin{conclusion}
  $\dim V = m$.
\end{conclusion}
%
\begin{definition}
  \emph{Układem fundamentalnym} nazwiemy dowolną bazę $V$.
\end{definition}
%
\begin{definition}
  \emph{Macierzą rozwiązującą} będziemy nazywali macierz, której kolumny tworzą układ fundamentalny.
\end{definition}
%
\begin{statement}
  $M(t)$ jest macierzą rozwiązującą wtedy i tylko wtedy, gdy jest nieosobliwa dla każdego (równoważnie pewnego)
  $t \in I$ oraz spełnia
%
  \begin{equation*}
    \frac{d}{dt} M(t) = A(t) \cdot M(t).
  \end{equation*}
\end{statement}
%
\begin{remark}
  Jeśli $M(t)$ jest macierzą rozwiązującą, a $P$ macierzą o~$m$~wierszach i stałych współczynnikach, to
%
  \begin{equation*}
    \frac{d}{dt} M(t) P = A(t) M(t) P.
  \end{equation*}
%
  Jeśli $P$ jest nieosobliwa, to $M(t)P$ jest macierzą rozwiązującą.
%
  \begin{equation*}
    \frac{d}{dt} M(t)P = \frac{d M(t)}{dt} P = A(t) M(t) P = A(t) \p[\big]{M(t) P}.
  \end{equation*}
%
  Jeśli $\mathcal{C} \in \R^m$, to $\mathcal{C} \cdot M(t)$ jest rozwiązaniem ogólnym.
  
  Jeśli $M(t)$ jest macierzą rozwiązującą, zaś $t_0 \in I$, to 
  \begin{equation*}
    M(t, t_0) = M(t) \cdot \big[ M(t_0) \big]^{-1}
  \end{equation*}
%
  jest macierzą rozwiązującą oraz $M(t_0,t_0) = E$. Ponadto $M(t_0,t_0) y_0$ jest rozwiązaniem równania z warunkiem 
  początkowym $y(t_0) = y_0$.
\end{remark}
































