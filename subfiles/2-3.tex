%!TEX root = ../RRZ.tex
%
\begin{definition}
  Równaniem liniowym nazywamy równanie postaci
  %
  \begin{equation*}
    \frac{dy}{dt} = A(t)y + B(t),
  \end{equation*}
  %
  gdzie $A(t)$ jest macierzą $m \times m$, a $B(t)$ wektorem z $\R^m$ o ciągłych współczynnikach, określonym na 
  przedziale otwartym $I \subset \R$.
\end{definition}
%
\begin{definition}
  Równanie $y' = A(t)y$ nazywamy \emph{jednorodnym}, a równanie $y' = A(t)y + B(t)$ (odpowiadającym) niejednorodnym.
\end{definition}

\begin{theorem}
  Zbiór rozwiązań równania jednorodnego jest podprzestrzenią liniową $C^0(I,\R^m)$, a zbiór rozwiązań równania 
  niejednorodnego jej warstwą.
\end{theorem}

\noindent Niech $V$ oznacza zbiór rozwiązań wysyconych równania $y' = A(t)y$.

%\begin{statement}
%  Następujące warunki są równoważne dla zbioru $\{y_1, \ldots, y_n\} \subset V$:
%  \begin{enumerate}
%    \item Zbiór jest liniowo niezależny.
%    \item Dla dowolnego $t \in I$ zbiór $\big\{y_1(t), \ldots, y_n(t)\big\}$ jest liniowo niezależny w $\R^m$.
%    \item Istnieje $t \in I$, że zbiór $\big\{y_1(t), \ldots, y_n(t)\big\}$ jest liniowo niezależny w $\R^m$.
%  \end{enumerate}
%\end{statement}