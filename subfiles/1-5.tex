%!TEX root = ../RRZ.tex
%
\begin{definition}
  \emph{Wykładnikiem Lapunowa} macierzy $A$ nazywamy liczbę
  %
  \begin{equation*}
    \overline{\lambda} = \max \{ \real \lambda_k : k = 1, \ldots, n \}.
  \end{equation*}
\end{definition}
%
\begin{definition}
  \emph{Potęgą Lapunowa} macierzy $A$ nazywamy liczbę
  %
  \begin{equation*}
    \overline{l} = \max \bigl\{ l \geq 0 : \exists k \in \{1,\ldots,n\} \quad \real \lambda_k = \overline{\lambda} 
    \ \wedge \ M_{k,l} \not= 0 \bigr\}.
  \end{equation*}
\end{definition}
%
\begin{lemma}[Wzór Leibniza]
  \begin{equation} \label{Leibniz}
    D^n(fg) = \sum_{k=0}^n \binom{n}{k} D^k f \cdot D^{n-k} g.
  \end{equation}
\end{lemma}
%
\begin{theorem}
  Dla każdej macierzy $A$ zachodzą nierówności:
%
  \begin{equation*}
    0 < \liminf_{t\to\infty} \frac{ \norm{e^{At}} }{ t^{\overline{l}} e^{\overline{\lambda} t} } \leq
    \limsup_{t\to\infty} \frac{ \norm{e^{At}} }{ t^{\overline{l}} e^{\overline{\lambda} t} } < \infty.
  \end{equation*}
\end{theorem}
%
\begin{proof}
  Nierówność środkowa jest oczywista. Zaczniemy wobec tego od prawej.
%
  \begin{align*}
    \limsup_{t\to\infty} \frac{ \norm{e^{At}} }{ t^{\overline{l}} e^{\overline{\lambda} t}} &=
    \limsup_{t\to\infty} \frac{\norm*{\sum\limits_{k=1}^n \sum\limits_{l=0}^{q_k - 1} M_{k,l} \cdot t^l e^{\lambda_k 
    t}}}{t^{\overline{l}} \exp(\overline{\lambda}t)} \\ &\leq
    \sum_{k=1}^n \sum_{l=0}^{q_k-1} \norm{M_{k,l}} \cdot \limsup_{t\to\infty} \frac{t^l e^{(\real \lambda_k) 
    t}}{t^{\overline{l}} e^{\overline{\lambda}t}} \\ &=
    \sum_{k=1}^n \sum_{l=0}^{q_k-1} \norm{M_{k,l}} \cdot \limsup_{t\to\infty} t^{l-\overline{l}} e^{(\real \lambda_k - 
    \overline{\lambda}) t}
  \end{align*}
%
  Zauważmy, że $\real \lambda_k - \overline{\lambda} \leq 0$, a jeśli $\real \lambda_k - \overline{\lambda} = 0$ oraz 
  $M_{k,l} \not= 0$, to $l \leq \overline{l}$. Wobec tego, dla każdej kombinacji $k$ i $l$ jest
%
  \begin{equation*}
    \limsup_{t\to\infty} t^{l-\overline{l}} \exp \p[\big]{(\real \lambda_k - \overline{\lambda}) t} \leq 1.
  \end{equation*}
%
  Bez utraty ogólności możemy przyjąć $\overline{\lambda} = \lambda_1$, $M_{1,\overline{l}} \not= 0$ oraz
%
  \begin{equation*}
    w(z) = \prod_{k=2}^{n} (z-\lambda_k)^{q_k} = \frac{\mychi_A(z)}{(z-\lambda_1)^{q_1}}.
  \end{equation*}
%
  Niech $f(z) = w(z) \cdot e^{zt}$. Wtedy dla $k>0$ oraz $l<q_k$ jest $f^{(l)}(\lambda_k) = 0$, bo pochodna jest sumą 
  członów ze wzoru \eqref{Leibniz}, gdzie $(z-\lambda_k)$ występuje w~potędze dodatniej. Niechaj teraz $k=1$ oraz 
  $\varphi(z) = e^{zt}$, $\psi(z) = w(z)$. Wtedy
%
  \begin{equation*}
    (\varphi \cdot \psi)^{(l)} (\lambda_1) = \sum_{i=0}^l \binom li t^i e^{\lambda_1 t} w^{(l-i)}(\lambda_i) =
    e^{\lambda_1 t} \cdot p_l(t),
  \end{equation*}
%
  gdzie $p_l$ jest wielomianem stopnia co najwyżej $l$. Wtedy $p_{\overline{l}}$ ma postać
%
  \begin{equation*}
    t^{\overline{l}} e^{\lambda_1 t} w(\lambda_1) + \widetilde{p}_{\overline{l}}(t) e^{\lambda_1 t},
  \end{equation*}
%
  gdzie stopień $\widetilde{p}_{\overline{l}}$ jest mniejszy od $\overline{l}$. Z twierdzenia spektralnego jest
%
  \begin{align*}
    \liminf_{t\to\infty} \frac{\norm[\big]{w(A) e^{At}}}{t^{\overline{l}} e^{\lambda_1 t}} &=
    \liminf_{t\to\infty} \frac{\norm[\bigg]{\sum\limits_{l=0}^{\overline{l}-1} M_{1,l} e^{\lambda_1 t} 
    p_l(t) + 
    M_{1,\overline{l}} e^{\lambda_1 t} \p[\big]{t^{\overline{l}} w(\lambda_1) + \widetilde{p}_{\overline{l}} (t)} }} 
    {t^{\overline{l}} e^{\lambda_1 t}} \\
    &= \liminf_{t\to\infty} \frac{\norm[\big]{e^{\lambda_1 t}} \cdot 
    \norm[\bigg]{\sum\limits_{l=0}^{\overline{l}-1} 
    M_{1,l} 
    p_l(t) + M_{1,\overline{l}} \p[\big]{t^{\overline{l}} w(\lambda_1) + \widetilde{p}_{\overline{l}} (t)} }} 
    {t^{\overline{l}} e^{\lambda_1 t}} \\
    &\geq \liminf_{t\to\infty} \frac{ \norm{M_{1,\overline{l}}} \cdot \abs[\big]{ t^{\overline{l}} w(\lambda_1) + 
    \widetilde{p}_{\overline{l}}(t) } - \sum\limits_{l=0}^{\overline{l}-1} \norm{M_{1,\overline{l}}} \cdot 
    \norm{p_l(t)} }{t^{\overline{l}}} \\
    &\geq \liminf_{t\to\infty} \norm{M_{1,\overline{l}}} \cdot \abs*{w(\lambda_1) + 
    \frac{\widetilde{p}_{\overline{l}}(t)}{t^{\overline{l}}}} - \limsup_{t\to\infty} \sum_{l=0}^{\overline{l}-1} 
    \norm{M_{1,l}} \cdot \abs*{\frac{p_l(t)}{t^{\overline{l}}}} \\ 
    &= \norm{M_{1,\overline{l}}} \cdot 
    \underbrace{\abs[\big]{w(\lambda_1)}}_{>0}.
  \end{align*}
%
  Ostatecznie
%
  \begin{equation*}
    \liminf_{t\to\infty} \frac{\norm[\big]{w(A) e^{At}}}{t^{\overline{l}} e^{\lambda_1 t}} \geq 
    \frac{1}{\norm[\big]{w(A)}} \cdot \liminf_{t\to\infty} \frac{\norm[\big]{w(A) e^{At}}}{t^{\overline{l}} 
    e^{\overline{\lambda} t}} \geq \frac{\norm{M_{1,\overline{l}}} \cdot \abs[\big]{w(\lambda_1)}}{\norm[\big]{w(A)}}
    >0. \qedhere
  \end{equation*}
\end{proof}
































