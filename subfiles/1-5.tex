%!TEX root = ../RRZ.tex
%
\begin{definition}
  \emph{Wykładnikiem Lapunowa} macierzy $A$ nazywamy liczbę
  %
  \begin{equation*}
    \overline{\lambda} = \max \{ \real \lambda_k : k = 1, \ldots, n \}.
  \end{equation*}
\end{definition}
%
\begin{definition}
  \emph{Potęgą Lapunowa} macierzy $A$ nazywamy liczbę
  %
  \begin{equation*}
    \overline{l} = \max \bigl\{ l \geq 0 : \exists k \in \{1,\ldots,n\} \quad \real \lambda_k = \overline{\lambda} 
    \ \wedge \ M_{k,l} \not= 0 \bigr\}.
  \end{equation*}
\end{definition}
%
\begin{theorem}
  Dla każdej macierzy $A$ zachodzą nierówności:
%
  \begin{equation*}
    0 < \liminf_{t\to\infty} \frac{\norm*{e^{At}}}{t^{\overline{l}} e^{\overline{\lambda} t}} \leq
    \limsup_{t\to\infty} \frac{\norm*{e^{At}}}{t^{\overline{l}} e^{\overline{\lambda} t}} < \infty.
  \end{equation*}
\end{theorem}
%
\begin{proof}
  Łatwe.
\end{proof}
































