%!TEX root = ../RRZ.tex

\noindent Niech
\begin{equation} \label{secondorder1}
  a_2(t)y'' + a_1(t)y' + a_0(t)y = 0,
\end{equation}
gdzie $a_2,a_1,a_0$ są analityczne w pewnym punkcie $t_0$.

\begin{definition}
  Powiemy, że $t_0$ jest \emph{punktem regularnym} wtedy i tylko wtedy, gdy $a_2(t_0) \not= 0$.
  W przeciwnym wypadku $t_0$ nazwiemy \emph{punktem osobliwym}.
\end{definition}

\noindent W przypadku regularnym równanie \eqref{secondorder2} sprowadza się do
\begin{equation} \label{secondorder2}
  y'' + p(t)y' + q(t)y = 0,
\end{equation}
gdzie $p$ i $q$ są analityczne w punkcie $t_0$, czyli
\[
  p(t) = \sum_{n=0}^{\infty}p_n(t - t_0)^n, \qquad q(t) = \sum_{n=0}^{\infty}q_n(t - t_0)^n.
\]

\begin{theorem}
  Każde rozwiązanie równania \eqref{secondorder2} jest analityczne w kole, w którym oba szeregi
  $p(t)$ i $q(t)$ zbiegają. Co więcej, analityczna funkcja
  \[
    y(t) = \sum_{n=0}^{\infty}c_n(t - t_0)^n
  \]
  jest rozwiązaniem wtedy i tylko wtedy, gdy
  \[
    c_{n+2} = - \frac{1}{(n+1)(n+2)} \p*{ \sum_{k=0}^{n} c_{k+1}(k+1)p_{n-k} + \sum_{k=0}^{n} c_k + q_{n-k} }
  \]
\end{theorem}