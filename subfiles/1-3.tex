%!TEX root = ../RRZ.tex

Rozważmy równanie
\begin{equation} \label{secondorder1}
  a_2(t) y'' + a_1(t)y' + a_0(t)y = 0,
\end{equation}
gdzie $a_2,a_1,a_0$ są analityczne w pewnym punkcie $t_0$.
\begin{definition}
  Powiemy, że $t_0$ jest \emph{punktem regularnym} wtedy i tylko wtedy, gdy $a_2(t_0) \not= 0$. W przeciwnym wypadku 
  $t_0$ nazwiemy \emph{punktem osobliwym}.
\end{definition}
W przypadku regularnym równanie \eqref{secondorder2} sprowadza się do
\begin{equation} \label{secondorder2}
  y'' + p(t)y' + q(t)y = 0,
\end{equation}
gdzie $p$ i $q$ są analityczne w punkcie $t_0$, czyli
\[
  p(t) = \sum_{n=0}^{\infty} p_n(t-t_0)^n, \qquad q(t) = \sum_{n=0}^{\infty} 
  q_n(t-t_0)^n.
\]
\begin{theorem}
  Każde rozwiązanie równania \eqref{secondorder2} jest analityczne w kole, w którym oba szeregi $p(t)$ i $q(t)$ 
  zbiegają. Co więcej, analityczna funkcja
  \[
    y(t) = \sum_{n=0}^{\infty}c_n(t - t_0)^n
  \]
  jest rozwiązaniem wtedy i tylko wtedy, gdy
  \begin{equation} \label{rr}
    c_{n+2} = - \frac{1}{(n+1)(n+2)} \p*{ \sum_{k=0}^{n} c_{k+1}(k+1)p_{n-k} + \sum_{k=0}^{n} c_k q_{n-k} }.
  \end{equation}
\end{theorem}
\begin{proof}
  Dla ustalenia uwagi niech $t_0 = 0$ oraz $y(t) = \sum_{n=0}^{\infty} c_nt^n$. Wtedy
  \[
    y'(t) = \sum_{n=0}^{\infty} (n+1) c_{n+1} t^n, \qquad y''(t) = \sum_{n=0}^{\infty} (n+1)(n+2) c_{n+2} t^n.
  \]
  Z iloczynu Cauchy'ego%
  \footnote{
    $ \p*{ \sum_{n=0}^{\infty} a_n } \cdot \p*{ \sum_{n=0}^{\infty} b_n } =
    \sum_{n=0}^{\infty} \sum_{k=0}^{n} a_k b_{n-k} $
  }
  dostajemy
  \begin{flalign*}
    p(t) y'(t) &= \p*{ \sum_{n=0}^{\infty} p_n t^n} \cdot \p*{\sum_{n=0}^{\infty} (n+1) c_{n+1} t^n} 
        = \sum_{n=0}^{\infty} t^n \sum_{k=0}^{n} (k+1) c_{k+1} p_{n-k}, & \\
    q(t) y(t) &= \p*{ \sum_{n=0}^{\infty} q_n t^n } \cdot \p*{ \sum_{n=0}^{\infty} c_n t^n } 
        = \sum_{n=0}^{\infty} t^n \sum_{k=0}^{n} c_k q_{n-k}.
  \end{flalign*}
  Rozpisując lewą stronę równania \eqref{secondorder2}, otrzymujemy
  \[
    \sum_{n=0}^{\infty} t^n \p*{ (n+1)(n+2)c_{n+2} + \sum_{k=0}^{n} (k+1) c_{k+1} p_{n-k} + 
    \sum_{k=0}^{n} c_k q_{n-k} } = 0.
  \]
  Z analityczności, dla każdego $n \geq 0$ jest
  \begin{equation*}
    (n+1)(n+2)c_{n+2} + \sum_{k=0}^{n} (k+1) c_{k+1} p_{n-k} + \sum_{k=0}^{n} c_k q_{n-k} = 0,
  \end{equation*}
  co dowodzi wzoru \eqref{rr}.
  
  Wzór rekurencyjny \eqref{rr} zadaje współczynniki $c_n$ dla $n \geq 2$, jeśli wybrane zostały $c_0$, $c_1$. Zauważmy, 
  że $c_0=y(t_0)$, $c_1=y'(t_0)$. Zatem dobierając $c_0$ oraz $c_1$ możemy otrzymać dowolny warunek początkowy 
  dla $y$, co pozwala uzyskać każde rozwiązanie wysycone. Pozostaje pokazać, że przy dowolnym wyborze $c_0$, $c_1$ wzór 
  \eqref{rr} prowadzi do szeregu Taylora funkcji analitycznej w kole $D(t_0,R)$.
  
  Wybierzmy $0 < r < R$. Wtedy funkcje $p,q$ są zbieżne w $\overline{D} (t_0,r)$ oraz
  \begin{equation*}
    \sum_{n=0}^{\infty} \abs{p_n} r^n < \infty, \qquad \sum_{n=0}^{\infty} \abs{q_n} r^n < \infty.
  \end{equation*}
  Wobec tego istnieją stałe $L_p$, $L_q$ takie, że dla dowolnego $n \geq 0$ jest
  \begin{equation*}
    \abs{p_n} r^n \leq L_p, \qquad \abs{q_n} r^n \leq L_q.
  \end{equation*}
  Niech $0 < \rho < r$ oraz $\gamma_n = \abs{c_n} \rho^n$, $\Gamma_n = \max\{\gamma_j: j = 1,\ldots,n\}$. Wtedy
  \begin{align*}
    \abs{\gamma_{n+2}}
    &\leq \frac{\rho^{n+2}}{(n+1)(n+2)} \p*{ \sum_{k = 0}^{n} (k + 1) \cdot \abs{c_{k + 1}} \cdot \abs{p_{n - k}} + 
        \sum_{k = 0}^{n} \abs{c_k} \cdot \abs{q_{n - k}}} \\ 
    &\leq \frac{\rho^{n+2}}{(n+1)(n+2)} \p*{ \sum_{k=0}^{n} (n+1) \cdot \frac{\gamma_{k+1}}{\rho^{k+1}} 
        \cdot \frac{L_p}{r^{n-k}} + \sum_{k=0}^{n} \frac{\gamma_k}{\rho^k} \cdot \frac{L_q}{r^{n-k}} } \\ 
    &\leq \frac{\rho^{n+2}}{(n+1)(n+2)} \p*{ (n+1) \sum_{k=0}^{n} \frac{\Gamma_{n+1}}{\rho^{k-n} \rho^{n+1}} \cdot 
        \frac{L_p}{r^{n-k}} + \sum_{k=0}^{n} \frac{\Gamma_n}{\rho^{k-n}\rho^{n}} \cdot \frac{L_q}{r^{n-k}} } \\
    &\leq \frac{\rho L_p}{n+2} \Gamma_{n+1} \sum_{k=0}^{n} \p*{ \frac{\rho}{r} }^{n-k} +
        \frac{\rho^2 L_q}{(n+1)(n+2)} \Gamma_n \sum_{k=0}^{n} \p*{ \frac{\rho}{r} }^{n-k} \\
    &\leq \p*{ \frac{\rho L_p}{n+2} + \frac{\rho^2 L_q}{(n+2)(n+1)}} \Gamma_{n+1}
        \sum_{k=0}^n \p*{ \frac{\rho}{r} }^{n-k} \\
    &\leq \underbrace{\p*{ \frac{\rho L_p}{n+2} + \frac{\rho^2 L_q}{(n+2)(n+1)}} \cdot
        \p*{\frac{1}{1 - \frac{\rho}{r}}}}_{(\ast)} \Gamma_{n+1}.
  \end{align*}
  Wyrażenie $(\ast)$ zbiega do zera, gdy $n \to \infty$. Zauważmy, że
  \begin{equation*}
    \exists N \ \forall n \geq N \quad \gamma_{n+2} \leq \Gamma_{n+1} \implies \Gamma_{n+2} = 
    \Gamma_{n+1}.
  \end{equation*}
  Oznacza to, że ciąg $\Gamma_n$ jest stały od pewnego miejsca i ograniczony przez 
  pewne~$\overline{\Gamma}$. Wtedy $\abs{c_n}\rho^n \leq \overline{\Gamma}$. Zatem jeśli $\abs{t} < 
  \rho$, to z kryterium Cauchy'ego jest
  \begin{equation*}
    \sqrt[n]{\abs{c_n} \cdot \abs{t}^n} = \sqrt[n]{\abs{c_n} \cdot \rho^n} \cdot 
    \sqrt[n]{\frac{\abs{t}^n}{\rho^n}} \leq \sqrt[n]{\overline{\Gamma}} \cdot \abs*{\frac{t}{\rho}} 
    < 1.
  \end{equation*}
  Szereg $\sum c_n t^n$ jest zbieżny w kole o promieniu $\rho$. Ponieważ $\rho$ może być dowolnie 
  bliskie $R$, to suma kół wypełnia koło otwarte o promieniu $R$.
\end{proof}


































