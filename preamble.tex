\documentclass{mwrep}

\usepackage[utf8]{inputenc}
\usepackage[T1, nomathsymbols]{polski}
\usepackage{lmodern}
\usepackage{amsmath, amsthm, amssymb, amsfonts}
\usepackage{mathtools}
\usepackage{thmtools}
\usepackage[
  paperheight = 297mm,
  paperwidth  = 210mm,
  top         = 2.5cm,
  bottom      = 2.5cm,
  left        = 4.4cm,
  right       = 4.4cm
]{geometry}

\declaretheoremstyle[
  spaceabove = \topsep,
  spacebelow = \topsep,
  headfont   = \bfseries,
  notefont   = \bfseries,
  notebraces = {(}{)},
  bodyfont   = \normalfont,
]{mystyle}

\let\proof\relax

\declaretheorem[
  style  = mystyle,
  name   = Twierdzenie,
  parent = section
]{theorem}
\declaretheorem[
  style   = mystyle,
  name    = Lemat,
  sibling = theorem
]{lemma}
\declaretheorem[
  style   = mystyle,
  name    = Definicja,
  sibling = theorem
]{definition}
\declaretheorem[
  style   = mystyle,
  name    = Wniosek,
  sibling = theorem
]{conclusion}
\declaretheorem[
  style   = mystyle,
  name    = Przykład,
  sibling = theorem
]{example}
\declaretheorem[
  style    = mystyle,
  name     = Dowód,
  numbered = no,
  qed      = \ensuremath{\blacksquare}
]{proof}

\newcommand*{\R}{\mathbb{R}}
\newcommand*{\N}{\mathbb{N}}
\newcommand*{\Q}{\mathbb{Q}}
\newcommand*{\Z}{\mathbb{Z}}
\newcommand*{\C}{\mathbb{C}}

\DeclarePairedDelimiter\abs{\lvert}{\rvert}
\DeclarePairedDelimiter\norm{\lVert}{\rVert}
\DeclarePairedDelimiter\p{(}{)}

\DeclareMathOperator{\Dm}{Dm}
\DeclareMathOperator{\spectre}{sp}

\newcommand{\goodchi}{\protect\raisebox{2pt}{$\chi$}}

\pagestyle{uheadings}
\makeatletter
\renewcommand\heading@font\scshape
\makeatother

\title{
  \huge \textbf{Równania różniczkowe zwyczajne} \\
  \Large Opracowanie zagadnień na egzamin
}
\author{ KJG }
\date{ Wersja z \today }


















