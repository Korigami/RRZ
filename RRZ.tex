\documentclass{mwrep}

\usepackage[utf8]{inputenc}
\usepackage[nomathsymbols]{polski}
\usepackage{amsmath, amsthm, amssymb, amsfonts}
\usepackage{mathtools}
\usepackage{thmtools}
\usepackage[
  paperheight = 297mm,
  paperwidth  = 210mm,
  top         = 2.5cm,
  bottom      = 2.5cm,
  left        = 4.4cm,
  right       = 4.4cm
]{geometry}

\declaretheoremstyle[
  spaceabove = \topsep,
  spacebelow = \topsep,
  headfont   = \bfseries \scshape,
  notefont   = \normalfont,
  notebraces = {(}{)},
  bodyfont   = \normalfont,
]{mystyle}

\let\proof\relax

\declaretheorem[
  style  = mystyle,
  name   = Twierdzenie,
  parent = section
]{theorem}
\declaretheorem[
  style   = mystyle,
  name    = Lemat,
  sibling = theorem
]{lemma}
\declaretheorem[
  style   = mystyle,
  name    = Definicja,
  sibling = theorem
]{definition}
\declaretheorem[
  style   = mystyle,
  name    = Wniosek,
  sibling = theorem
]{conclusion}
\declaretheorem[
  style   = mystyle,
  name    = Przykład,
  sibling = theorem
]{example}
\declaretheorem[
  style    = mystyle,
  name     = Dowód,
  numbered = no,
  qed      = \ensuremath{\blacksquare}
]{proof}

\newcommand*{\R}{\mathbb{R}}
\newcommand*{\N}{\mathbb{N}}
\newcommand*{\Q}{\mathbb{Q}}
\newcommand*{\Z}{\mathbb{Z}}
\newcommand*{\C}{\mathbb{C}}

\DeclarePairedDelimiter\abs{\lvert}{\rvert}
\DeclarePairedDelimiter\norm{\lVert}{\rVert}
\DeclarePairedDelimiter\p{(}{)}

\pagestyle{uheadings}
\makeatletter
\renewcommand\heading@font\scshape
\makeatother

\title{
  \huge \textbf{Równania różniczkowe zwyczajne} \\
  \Large Opracowanie zagadnień na egzamin
}
\author{ KJG }
\date{ Wersja z \today }

\begin{document}
	
  \maketitle
  
  \tableofcontents
  
  \chapter{Twierdzenia}
%    \section{Ciągła zależność od parametru}
%    \section{Różniczkowalna zależność od parametru}
%    \section{Rozwiązania przez szeregi potęgowe wokół punktu regularnego}
%    \section{Twierdzenie spektralne dla funkcji analitycznych}
%    \section{Twierdzenie o asymptotycznym zachowaniu $\mathbf{ \norm*{e^{At}} }$}
%    \section{Twierdzenie o minimach funkcji Lapunowa i stabilności}
  
  \chapter{Zagadnienia}
    \section{Istnienie i jednoznaczność rozwiązań, rozwiązania wysycone}
      \begin{theorem}[Peano]
  Niech $y' = f(y,t)$, gdzie $y(t_0) = y_0$ oraz
  \[
    f \colon H = \overline{B} (y_0, b) \times [t_0 - a, t_0 + a] \longrightarrow \R^m.
  \]
  Załóżmy, że funkcja $f$ jest ciągła i oznaczmy
  \[
    M = \sup \big\{ \norm[\big]{f(y,t)} : (y, t) \in H \big\}.
  \]
  Wówczas dla $\alpha = \min(a, b/M)$ istnieje rozwiązanie $y(t)$ określone na
  przedziale $[t_0 - \alpha, t_0 + \alpha]$, spełniające warunek początkowy $y(t_0) = y_0$.
\end{theorem}

\begin{theorem}[Picard — Lindelöf]
  Niech $y' = f(y,t)$, gdzie $y(t_0) = y_0$ oraz
  \[
    f \colon H = \overline{B} (y_0, b) \times [t_0 - a, t_0 + a] \longrightarrow \R^m.
  \]
  Załóżmy, że funkcja $f$ jest ciągła oraz lipszycowska ze względu na $y$, to znaczy
  \[
    \exists L \ \forall (y_1,t), (y_2,t) \in H \quad
    \norm[\big]{f(y_1,t) - f(y_2,t)} \leq L \cdot \norm{y_1 - y_2}.
  \]
  Oznaczmy ponadto
  \[
    M = \sup \big\{ \norm[\big]{f(y,t)} : (y, t) \in H \big\}.
  \]
  Wówczas dla dowolnego $\alpha < \min(a, b/M, 1/L)$ istnieje dokładnie jedno rozwiązanie
  zagadnienia Cauchy'ego z warunkiem początkowym $y(t_0) = y_0$ określone na przedziale
  $[t_0 - \alpha, t_0 + \alpha]$.
\end{theorem}

\begin{proof}
  Jako ćwiczenie.
\end{proof}
%    \section{Metoda Frobeniusa}
%    \section{Rozwiązania układów liniowych jednorodnych}
%    \section{Rozwiązywanie równań liniowych niejednorodnych}
%    \section{Hiperboliczność i stabilność punktów równowagi}
%    \section{Zagadnienia brzegowe}
  
  \chapter{Przykłady}
%    \section{Rozwiązywanie równań metodą szeregów potęgowych}
%    \section{Równania na wariację}
%    \section{Potoki -- policzenie i zastosowanie własności w konkretnych sytuacjach}
%    \section{Wzory Liouville'a i Abela}
%    \section{Zastosowania twierdzenia spektralnego, macierze spektralne}
%    \section{Całki pierwsze, funkcje Lapunowa -- zastosowanie do badania stabilności}

\end{document}













