%!TEX root = ../RRZ.tex

\documentclass{mwrep}

\usepackage[utf8]{inputenc}

\usepackage[T1, nomathsymbols]{polski}

\usepackage{lmodern}

\usepackage{amsmath, amsthm, amssymb, amsfonts}
    \numberwithin{equation}{section}

\usepackage{mathtools}
    \DeclarePairedDelimiter\abs{\lvert}{\rvert}
    \DeclarePairedDelimiter\norm{\lVert}{\rVert}
    \DeclarePairedDelimiter\p{(}{)}
    \DeclareMathOperator{\Domain}{Dm}
    \DeclareMathOperator{\spectre}{sp}

\usepackage{thmtools}
    \declaretheoremstyle[
      spaceabove = \topsep,
      spacebelow = \topsep,
      headfont   = \bfseries,
      notefont   = \bfseries,
      notebraces = {(}{)},
      bodyfont   = \normalfont,
    ]{mystyle}
    
    \let\proof\relax
    
    \declaretheorem[style = mystyle, name = Twierdzenie,   parent  = section]{theorem}
    \declaretheorem[style = mystyle, name = Stwierdzenie, sibling  = theorem]{statement}
    \declaretheorem[style = mystyle, name = Lemat,        sibling  = theorem]{lemma}
    \declaretheorem[style = mystyle, name = Definicja,    sibling  = theorem]{definition}
    \declaretheorem[style = mystyle, name = Wniosek,      sibling  = theorem]{conclusion}
    \declaretheorem[style = mystyle, name = Przykład,     sibling  = theorem]{example}
    \declaretheorem[style = mystyle, name = Dowód,        numbered = no, qed = \ensuremath{\blacksquare}]{proof}
    \declaretheorem[style = mystyle, name = Dowód,        numbered = no, qed = \ensuremath{\square}]{nestedproof}

\usepackage{parskip}

\newcommand*{\R}{\mathbb{R}}
\newcommand*{\N}{\mathbb{N}}
\newcommand*{\Q}{\mathbb{Q}}
\newcommand*{\Z}{\mathbb{Z}}
\newcommand*{\C}{\mathbb{C}}
\newcommand*{\mychi}{\protect\raisebox{2pt}{$\chi$}}

\pagestyle{uheadings}
\makeatletter
\renewcommand\heading@font\scshape
\makeatother

\title{
  \huge \textbf{Równania różniczkowe zwyczajne} \\
  \Large Opracowanie zagadnień na egzamin
}
\author{ KJG }
\date{ Wersja z \today }




































\begin{document}

  \maketitle
  \newpage~
  \thispagestyle{empty}
  \tableofcontents
  
  \chapter{Twierdzenia}
    \section{Ciągła zależność od parametrów}
      %!TEX root = ../RRZ.tex

\begin{theorem}[o ciągłej zależności od parametru]
  Niech
  \[
    y' = f(y,t,\lambda), \qquad
    f \colon \R^{m + 1} \times \R^l \supset U \times B_l(\lambda_0,c) \longrightarrow \R^m,
  \]
  gdzie $f$ jest funkcją ciągłą oraz $c > 0$. Niech $y(t,\lambda_0)$ będzie rozwiązaniem równania
  $y' = f(y,t,\lambda_0)$ z warunkiem początkowym $(y_0, t_0)$ określonym na \emph{zwartym}
  przedziale $I$ zawierającym $t_0$. Wybierzmy $b > 0$ i rozważmy zbiór 
  \[
    R_b = \big\{ (y,t) : t \in I, \norm[\big]{ y - y(t,\lambda_0) } < b \big\}.
  \]
  Załóżmy dalej, że
  \begin{enumerate}
    \item Istnieje $L \geq 0$, że dla wszystkich $(y_1,t),(y_2,t) \in R_b$ jest
    \[
      \norm[\big]{ f(y_1,t,\lambda_0) - f(y_2,t,\lambda_0) } \leq L \cdot \norm{y_1 - y_2},
    \]
    \item Dla dowolnego $\varepsilon > 0$ istnieje $\delta > 0$, że dla $(y,t) \in R_b$ zachodzi
    \[
      \norm{\lambda - \lambda_0} < \delta \implies \norm[\big]{ f(y,t,\lambda) - f(y,t,\lambda_0) } < \varepsilon.
    \]
  \end{enumerate}
  Wówczas istnieje stała $c^{\ast} > 0$ taka, że
  \begin{enumerate}
    \item Jeśli $\norm{\lambda - \lambda_0} < c^{\ast}$, to $y(t,\lambda)$ jest określone na $I$,
    \item Jeśli $\lambda_n \to \lambda_0$, to $y(t,\lambda_n) \rightrightarrows y(t,\lambda_0)$ na $I$.
  \end{enumerate}
\end{theorem}
    \section{Różniczkowalna zależność od parametrów}
      
%!TEX root = ../RRZ.tex

	%
	\begin{theorem}[O różniczkowalnej zależności od parametru]
		Niech
		%
		\begin{equation*}
		y' = f(y,t,\lambda), \qquad
		f \colon \R^{m+1} \times \R \supset U \times (\lambda_0 - c, \lambda_0 + c) \longrightarrow \R^m,
		\end{equation*}
		%
		gdzie $f$ jest funkcją ciągłą względem $y,t,\lambda$ oraz klasy $C^1$ względem $y,\lambda$. Ustalmy warunek 
		początkowy $(y_0,t_0)$ i oznaczmy przez $y(t,\lambda)$ rozwiązanie równania
		%
		\begin{equation*}
		\frac{\partial y(t,\lambda)}{\partial t} = f(y,t,\lambda)
		\end{equation*}
		%
		z warunkiem początkowym $y(t_0,\lambda) = y_0$, określone na ustalonym i \emph{zwartym} przedziale $I$. Wówczas na 
		przedziale $I$ istnieje ciągła funkcja
		%
		\begin{equation*}
		z(t,\lambda_0) = \frac{\partial y(t,\lambda)}{\partial \lambda} \Big\vert_{\lambda = \lambda_0}
		\end{equation*}
		%
		oraz zachodzi równość
		%
		\begin{equation*}
		\frac{\partial z(t,\lambda_0)}{\partial t} =
		\frac{\partial^2 y(t,\lambda)}{\partial t \partial \lambda} \Big\vert_{\lambda = \lambda_0} =
		\frac{\partial^2 y(t,\lambda)}{\partial \lambda \partial t} \Big\vert_{\lambda = \lambda_0}.
		\end{equation*}
	\end{theorem}
	
	\begin{proof}
		%
		Niech $y_\l(t) \in R_b$. Niech $w_\l(t) \coloneqq \frac{ y_\l(t) - y_{\l_0}(t) }{ \l - \l_0 }$ dla $\l \ne \l_0$.
		Wtedy:
		%
		\begin{align*}
		%
		\pd{t} w_\l(t) &= \frac{ \pd{t} y_\l(t) - \pd{t} y_{\l_0}(t) }{ \l - \l_0 } = \frac{ f_\l(y_\l(t), t) - 
			f_{\l_0}(y_{\l_0}(t), t) }{ \l - \l_0 } = \\
		&=  \frac{ f_\l(y_{\l_0}(t) + (\l - \l_0)w_\l(t), t) - 
			f_{\l_0}(y_{\l_0}(t), t) }{ \l - \l_0 } \text{.}
		%
		\end{align*}
		%
		Rozważmy dalej funkcję 
		%
		\begin{equation*}
		%
		F_\l(w, t) \coloneqq \frac{ f_\l(y_{\l_0}(t) + (\l - \l_0)w, t) -  f_{\l_0}(y_{\l_0}(t), t) }{ \l - \l_0 } \text{.}
		%
		\end{equation*}
		%
		(Zwróćmy uwagę na to, że w definicji $F$ $w$ jest symbolem argumentu, a nie funkcji.) 
		
		Pokażemy, że istnieje taka $c* > 0$, że o ile $ \abs{ \l - \l_0} \le c*$, to dziedzina $F$ dla każdego $\l$ spełniającego ten warunek jest zbiorem zwartym, a także, że $w_\l(t)$ zawiera się w tej dziedzinie dla każdego $t \in I$. 
		
		Z lematu udowodnionego w Twierdzeniu o ciągłej zależności od parametru wiemy, że istnieją takie stałe $c_1, K_1 > 0$ takie, że o ile $\abs{ \l - \l_0 } < c_1$, to $\norm{ w_\l(t) } \le K_1$. Z kolei, jeśli $\abs{ \l - \l_0 } \le \frac{ b }{ 2K_1 }$, 
		to dla $w : \norm{ w } \le 2K_1$ zachodzi:
		%
		\begin{equation*}
		%
		\norm{ y_{\l_0}(t) + (\l - \l_0)w - y_{\l_0}(t) } = \norm { (\l - \l_0)w  } = \abs{ \l - \l_0 } \norm{ w } \le
		\frac{ b }{ 2K_1 } 2K_1 = b \text{,}
		%
		\end{equation*}
		%
		co oznacza, że $( y_{\l_0}(t) + (\l - \l_0)w, t) \in R_b$ dla każdego $t \in I$. Oznaczmy $c^* = min(c_1, \frac{ b }{ 2K_1 } )$. Dzięki temu $F$ jest prawidłowo zdefiniowana na zbiorze
		%
		\begin{equation*}
		%
		\overline{B}(0, 2K_1) \times I \times ( (\l_0 - c^*, \l_0 + c^*) \setminus \{ \l_0 \} ) \text{.}
		%
		\end{equation*}
		%
		Dodefiniujemy $F$ dla $\l = \l_0$ tak, by była ona ciągła względem $\l$ w tym punkcie. Zachodzi:
		%
		\begin{align*}
		%
		\lim_{\l \to \l_0} F_\l(w, t) &= \lim_{\l \to \l_0} \frac{ f_\l(y_{\l_0}(t) + (\l - \l_0)w, t) -  f_{\l_0}(y_{\l_0}(t), t) }{ \l - \l_0 } = \\
		&=  \lim_{\l \to \l_0} \frac{ f_\l(y_{\l_0}(t) + (\l - \l_0)w, t) -  f_\l (y_{\l_0}(t), t) }{ w(\l - \l_0) } \cdot w + \\
		&+ \lim_{\l \to \l_0} \frac{ f_\l(y_{\l_0}(t), t) -  f_{\l_0}(y_{\l_0}(t) }{ \l - \l_0 } = \\
		&= \pd{ y } f_{\l_0} (y, t) \, \Big\vert_{ y = y_{\l_0}(t) } \cdot w + \pd{ \l } f_{\l}( y_{\l_0}(t), t) \, \Big\vert_{ \l = \l_0 } \text{,}
		\end{align*}
		%
		przy czym powyższe pochodne istnieją, gdyż z założenia $f$ jest $C^1$ względem $y$ i $\l$. Niech $F_{\l_0}(w, t) \coloneqq \lim_{\l \to \l_0} F_\l(w, t)$. Wtedy $F$ jest zdefiniowana na zbiorze $\overline{B}(0, 2K_1) \times I \times (\l_0 - c^*, \l_0 + c^*)$, który jest zwarty dla każdego ustalonego $\l$.
		%
		\begin{nestedlemma}
			%
			Funkcja $F$ zdefiniowana powyżej spełnia założenia Twierdzenia o ciągłej zależności od parametru przy równaniu 
			różniczkowym $\pd{t} w = F_\l(w, t)$. (Jest on przyjmowany bez dowodu)
			%
		\end{nestedlemma}		
		%
		Z powyższego oraz na mocy Twierdzenia Picarda dla każdego $\l \in  (\l_0 - c*, \l_0 + c*)$ istnieje rozwiązanie
		powyższego równania różniczkowego a także istnieje jednoznaczne rozwiązanie dla $\l = \l_0$ określone
		w pewnym $J \subset I$ (BSO niech $J$ będzie maksymalny możliwy). Co więcej, z Twierdzenia
		o ciągłej zależności mamy, ze dla każdego ciągu $\l_n \to \l_0$ zachodzi $w_{\l_n}(t) \rightrightarrows w_{\l_0}(t)$.
		Stąd:
		%
		\begin{equation*}
		%
		w_{\l_0}(t) = \lim_{n \to \infty} w_{\l_n}(t) = \lim_{\l \to \l_0} w_{\l}(t) = \lim_{\l \to \l_0} \frac{ y_\l(t) - y_{\l_0}(t) }{ \l - \l_0 } = \pd{ \l } y_\l(t) \Big\vert_{ \l = \l_0 } \text{.}
		%
		\end{equation*}
		%
		Zatem $w_{\l_0}$ jest szukaną funkcją $z$ z tezy twierdzenia.
		
		Równość w twierdzeniu zachodzi, gdyz:
		%
		\begin{align*}
		%
		\pd{t} w_{\l_0}(t) &= \pd{t} \lim_{\l \to \l_0}  \frac{ y_\l(t) - y_{\l_0}(t) }{ \l - \l_0 } = \pd{t} \lim_{n \to \infty}
		\frac{ y_{\l_n}(t) - y_{\l_0}(t) }{ \l_n - \l_0 } = \\
		&= \lim_{n \to \infty} \frac{  \pd{t} y_{\l_n}(t) -  \pd{t} y_{\l_0}(t) }{ \l_n - \l_0 } = \lim_{\l \to \l_0} \frac{  \pd{t} y_{\l_n}(t) -  \pd{t} y_{\l_0}(t) }{ \l_n - \l_0 } = \\
		& = \pd{\l } \pd{t} y_\l (t) \Big\vert_{ \l = \l_0 } \text{.}
		%
		\end{align*}
		%
		Czyli istotnie $\pd{t} w_{\l_0}(t) = \pd{\l } \pd{t} y_\l (t) \Big\vert_{ \l = \l_0 }$.
		%
	\end{proof}



























    \section{Rozwiązania przez szeregi potęgowe wokół punktu regularnego}
      %!TEX root = ../RRZ.tex

\noindent Rozważmy równanie
\begin{equation} \label{secondorder1}
  a_2(t)y'' + a_1(t)y' + a_0(t)y = 0,
\end{equation}
gdzie $a_2,a_1,a_0$ są analityczne w pewnym punkcie $t_0$.

\begin{definition}
  Powiemy, że $t_0$ jest \emph{punktem regularnym} wtedy i tylko wtedy, gdy $a_2(t_0) \not= 0$.
  W przeciwnym wypadku $t_0$ nazwiemy \emph{punktem osobliwym}.
\end{definition}

\noindent W przypadku regularnym równanie \eqref{secondorder2} sprowadza się do
\begin{equation} \label{secondorder2}
  y'' + p(t)y' + q(t)y = 0,
\end{equation}
gdzie $p$ i $q$ są analityczne w punkcie $t_0$, czyli
\begin{equation*}
  p(t) = \sum_{n=0}^{\infty}p_n(t - t_0)^n, \qquad q(t) = \sum_{n=0}^{\infty}q_n(t - t_0)^n.
\end{equation*}

\begin{theorem}
  Każde rozwiązanie równania \eqref{secondorder2} jest analityczne w kole, w którym oba szeregi
  $p(t)$ i $q(t)$ zbiegają. Co więcej, analityczna funkcja
  \begin{equation*}
    y(t) = \sum_{n=0}^{\infty}c_n(t - t_0)^n
  \end{equation*}
  jest rozwiązaniem wtedy i tylko wtedy, gdy
  \begin{equation} \label{rr}
    c_{n+2} =
    - \frac{1}{(n+1)(n+2)} \p*{ \sum_{k=0}^{n} c_{k+1}(k+1)p_{n-k} + \sum_{k=0}^{n} c_k q_{n-k} }.
  \end{equation}
\end{theorem}

\begin{proof}
  Dla ustalenia uwagi niech $t_0 = 0$ oraz $y(t) = \sum_{n=0}^{\infty} c_nt^n$. Wtedy
  \begin{equation*}
    y'(t) = \sum_{n=0}^{\infty} (n+1) c_{n+1} t^n, \qquad
    y''(t) = \sum_{n=0}^{\infty} (n+1)(n+2) c_{n+2} t^n.
  \end{equation*}
  Z iloczynu Cauchy'ego\footnote{
    $ \p*{ \sum_{n=0}^{\infty} a_n } \cdot \p*{ \sum_{n=0}^{\infty} b_n } =
    \sum_{n=0}^{\infty} \sum_{k=0}^{n} a_k b_{n-k} $
  }
  dostajemy
  \begin{flalign*}
    p(t) y'(t) &= \p*{ \sum_{n=0}^{\infty} p_n t^n} \cdot \p*{\sum_{n=0}^{\infty} (n+1) c_{n+1} 
    t^n} = \sum_{n=0}^{\infty} t^n \sum_{k=0}^{n} (k+1) c_{k+1} p_{n-k}, & \\
    q(t) y(t) &= \p*{ \sum_{n=0}^{\infty} q_n t^n } \cdot \p*{ \sum_{n=0}^{\infty} c_n t^n } = 
    \sum_{n=0}^{\infty} t^n \sum_{k=0}^{n} c_k q_{n-k}.
  \end{flalign*}
  Rozpisując lewą stronę równania \eqref{secondorder2}, dostajemy
  \begin{equation*}
    \sum_{n=0}^{\infty} t^n \p*{ (n+1)(n+2)c_{n+2} + \sum_{k=0}^{n} (k+1) c_{k+1} p_{n-k} + 
    \sum_{k=0}^{n} c_k q_{n-k} } = 0.
  \end{equation*}
  Z analityczności, dla każdego $n \geq 0$ jest
  \begin{equation*}
    (n+1)(n+2)c_{n+2} + \sum_{k=0}^{n} (k+1) c_{k+1} p_{n-k} + \sum_{k=0}^{n} c_k q_{n-k} = 0,
  \end{equation*}
  co dowodzi wzoru \eqref{rr}.
  
  Relacja rekurencyjna \eqref{rr} zadaje współczynniki $c_n$ dla $n \geq 2$, jeśli 
  wybrane zostały $c_0$, $c_1$. Zauważmy, że $c_0 = y(t_0)$, $c_1 = y'(t_0)$. Zatem dobierając 
  $c_0$ oraz $c_1$ możemy otrzymać dowolny warunek początkowy dla $y$, co pozwala uzyskać każde 
  rozwiązanie wysycone. Pozostaje pokazać, że przy dowolnym wyborze $c_0$, $c_1$ relacja \eqref{rr} 
  prowadzi do szeregu Taylora funkcji analitycznej w kole $D(t_0,R)$. 
  
  Wybierzmy $0 < r < R$. Wtedy funkcje $p,q$ są zbieżne w $\overline{D} (t_0,r)$ oraz
  \begin{equation*}
    \sum_{n=0}^{\infty} \abs{p_n} r^n < \infty, \qquad \sum_{n=0}^{\infty} \abs{q_n} r^n < \infty.
  \end{equation*}
  Z powyższego, istnieją stałe $L_p$, $L_q$ takie, że dla dowolnego $n \geq 0$ jest
  \begin{equation*}
    \abs{p_n} r^n \leq L_p, \qquad \abs{q_n} r^n \leq L_q.
  \end{equation*}
  Niech $0 < \rho < r$ oraz $\gamma_n = \abs{c_n} \rho^n$, $\Gamma_n = \max\{\gamma_j: j = 
  1,\ldots,n\}$. Wtedy
  \begin{align*}
    \abs{\gamma_{n+2}} &\leq \frac{\rho^{n+2}}{(n+1)(n+2)}
    \p*{ \sum_{k = 0}^{n} (k + 1) \cdot \abs{c_{k + 1}} \cdot \abs{p_{n - k}} + 
    \sum_{k = 0}^{n} \abs{c_k} \cdot \abs{q_{n - k}}} \\ &\leq
    \frac{\rho^{n+2}}{(n+1)(n+2)}
    \p*{ \sum_{k=0}^{n} (n+1) \cdot \frac{\gamma_{k+1}}{\rho^{k+1}} \cdot \frac{L_p}{r^{n-k}} + 
    \sum_{k=0}^{n} \frac{\gamma_k}{\rho^k} \cdot \frac{L_q}{r^{n-k}} } \\ &\leq 
    \frac{\rho^{n+2}}{(n+1)(n+2)}
    \p*{ (n+1) \sum_{k=0}^{n} \frac{\Gamma_{n+1}}{\rho^{k-n} \rho^{n+1}} \cdot 
    \frac{L_p}{r^{n-k}} +
    \sum_{k=0}^{n} \frac{\Gamma_n}{\rho^{k-n}\rho^{n}} \cdot \frac{L_q}{r^{n-k}} } \\ &\leq
    \frac{\rho L_p}{n+2} \Gamma_{n+1} \sum_{k=0}^{n} \p*{ \frac{\rho}{r} }^{n-k} + 
    \frac{\rho^2 L_q}{(n+1)(n+2)} \Gamma_n \sum_{k=0}^{n} \p*{ \frac{\rho}{r} }^{n-k} \\ &\leq
    \p*{ \frac{\rho L_p}{n+2} + \frac{\rho^2 L_q}{(n+2)(n+1)}} \Gamma_{n+1}
    \sum_{k=0}^n \p*{ \frac{\rho}{r} }^{n-k} \\ &\leq
    \underbrace{\p*{ \frac{\rho L_p}{n+2} + \frac{\rho^2 L_q}{(n+2)(n+1)}} \cdot \p*{\frac{1}{1 - 
    \frac{\rho}{r}}}}_{(\ast)} \Gamma_{n+1}.
  \end{align*}
  Wyrażenie $(\ast)$ zbiega do zera, gdy $n \to \infty$. Zauważmy, że
  \begin{equation*}
    \exists N \ \forall n \geq N \quad \gamma_{n+2} \leq \Gamma_{n+1} \implies \Gamma_{n+2} = 
    \Gamma_{n+1}.
  \end{equation*}
  Oznacza to, że ciąg $\Gamma_n$ jest stały od pewnego miejsca i ograniczony przez 
  pewne~$\overline{\Gamma}$. Wtedy $\abs{c_n}\rho^n \leq \overline{\Gamma}$. Zatem jeśli $\abs{t} < 
  \rho$, to z kryterium Cauchy'ego jest
  \begin{equation*}
    \sqrt[n]{\abs{c_n} \cdot \abs{t}^n} = \sqrt[n]{\abs{c_n} \cdot \rho^n} \cdot 
    \sqrt[n]{\frac{\abs{t}^n}{\rho^n}} \leq \sqrt[n]{\overline{\Gamma}} \cdot \abs*{\frac{t}{\rho}} 
    < 1.
  \end{equation*}
  Szereg $\sum c_n t^n$ jest zbieżny w kole o promieniu $\rho$. Ponieważ $\rho$ może być dowolnie 
  bliskie $R$, to suma kół wypełnia koło otwarte o promieniu $R$.
\end{proof}



































    \section{Twierdzenie spektralne dla funkcji analitycznych}
      %!TEX root = ../RRZ.tex
%
\begin{definition}
  \emph{Widmem} macierzy $A$ nazywamy zbiór jej wartości własnych wraz z krotnościami i oznaczamy $\spectrum (A)$.
\end{definition}
%
\begin{theorem}[Hamilton-Cayley] \label{T: H-C}
  Dla każdej macierzy $A$ zachodzi $\mychi_A(A) = 0$, gdzie $\mychi_A$ jest wielomianem charakterystycznym macierzy $A$.
\end{theorem}
%
\begin{theorem}[Spektralne dla wielomianów]
  Niech $A$ będzie macierzą o wartościach własnych $\lambda_1,\ldots,\lambda_n$ z krotnościami $q_1,\ldots,q_n$. Wtedy 
  istnieją macierze $M_{k,l}$ dla $1 \leq k \leq n$, $0 \leq l \leq q_k-1$, zwane \emph{spektralnymi} takie, że dla 
  każdego wielomianu $f$ zachodzi:
%
  \begin{equation*}
    f(A) = \sum_{k=1}^{n} \sum_{l=0}^{q_k-1} M_{k,l} \cdot f^{(l)}(\lambda_k).
  \end{equation*}
\end{theorem}
%
\begin{proof}
  W celu udowodnienia twierdzenia będzie potrzebny lemat pomocniczy.
%
  \begin{nestedlemma} \label{L: 1.4.2}
    Macierze $M_{k,l}$ są jednoznacznie wyznaczone przez tezę twierdzenia spektralnego dla
    wielomianów postaci $f(z) = z^r$, gdzie $r = 0, \ldots, m-1$, a~$m$ jest stopniem wielomianu $f$.
  \end{nestedlemma}
%
  \begin{nestedproof}
    Otrzymujemy układ równań z niewiadomymi $M_{k,l}$, czyli
%   
    \begin{equation*}
      A^r = \sum_{k=1}^n \sum_{l=0}^{q_k-1} M_{k,l} \cdot r (r-1) \cdots (r-l+1) \cdot \lambda_k^{r-l}.
    \end{equation*}
%   
    Teza lematu oznacza, że układ ten jest oznaczony. Pokażemy liniową niezależność wierszy. Wybierzmy współczynniki 
    $c_r$ dla $r = 0, \ldots, m-1$, tak aby kombinacja liniowa wierszy z tymi współczynnikami wynosiła $0$, czyli dla
    każdych $k = 1,\ldots,n$ oraz $l = 0, \ldots, q_k-1$ jest
%   
    \begin{equation*}
      \sum_{r=0}^{m-1} c_r \cdot r (r-1) \cdots (r-l+1) \lambda_k^{r-l} = 0.
    \end{equation*}
%
    Rozważmy teraz wielomian
%    
    \begin{equation*}
      w(z) = \sum_{r=0}^{m-1} c_r z^r.
    \end{equation*}
%
    Otrzymaliśmy, że $w^{(l)}(\lambda_k) = 0$, czyli 
    $\lambda_k$ jest zerem z krotnością co najmniej $q_k$, a zatem suma krotności zer wielomianu $w$ jest równa co 
    najmniej $\sum_{k=1}^n q_k = m$, co jest sprzecznością, bo stopień wielomianu był co najwyżej $m-1$.
  \end{nestedproof}
%
  Twierdzenie zostanie udowodnione indukcyjnie ze względu na stopień $f$.
  
  Przypuśćmy, że twierdzenie zachodzi dla wielomianów stopnia mniejszego od $m+r$, gdzie $r \geq 0$. Z lematu 
  \ref{L: 1.4.2} teza zachodzi dla $r=0$. Zwróćmy uwagę, że obie strony twierdzenia są liniowe względem $f$. 
  Wystarczy więc pokazać je dla układu rozpinającego przestrzeń wielomianów stopnia mniejszego niż $m+r$. W~celu 
  pokazania, że twierdzenie zachodzi również dla wielomianów stopnia $m+r$, wystarczy pokazać dla $f_r(z) = z^r 
  \mychi_A(z)$, bo każdy wielomian
%
  \begin{equation*}
    f(z) = a_{m+r} z^{m+r} + \ldots + a_1 z + a_0
  \end{equation*}
%
  można zapisać jako
%
  \begin{equation*}
    f(z) = a_{m+r} f_r(z) + P(z),
  \end{equation*}
%
  gdzie $P$ jest wielomianem stopnia mniejszego niż $m+r$. Zauważmy, że
%
  \begin{equation*}
    L = f_r(A) = A^r \cdot \mychi_A(A) \overset{\ref{T: H-C}}{=} 0, \qquad 
    P = \sum_{k=1}^n \sum_{l=0}^{q_k-1} M_{k,l} \cdot f^{(l)}(\lambda_k),
  \end{equation*}
%
  a ponadto $f_r(z) = (z-\lambda_k)^{q_k} \cdot Q(z)$. Pochodne rzędu niższego od $q_k$ składają się z sum członów, w 
  których $(z-\lambda_k)$ występuje w dowolnej potędze, więc zerują się przy podstawieniu $z=\lambda_k$.
%
\end{proof}
%
\begin{theorem}[Spektralne dla funkcji analitycznych]
  Niech
%
  \begin{equation*}
    f(z) = \sum_{n=0}^{\infty} a_n z^n, \quad \abs{z} < R.
  \end{equation*}
%
  Załóżmy, że $\spectrum(A) \subset D(0,R)$. Wówczas szereg $f(A)$ zbiega i zachodzi teza twierdzenia spektralnego dla 
  wielomianów:
%
  \begin{equation*}
    f(A) = \sum_{k=1}^{n} \sum_{l=0}^{q_k-1} M_{k,l} \cdot f^{(l)}(\lambda_j).
  \end{equation*}
\end{theorem}
%
\begin{proof}
  Oznaczmy
%
  \begin{equation*}
    f_N(z) = \sum_{n=0}^N a_n z^n.
  \end{equation*}
%
  Korzystając z twierdzenia spektralnego dla wielomianów, dostajemy
%
  \begin{multline*}
    f(A) = \lim_{N\to\infty} f_N(A) 
    = \lim_{N\to\infty} \sum_{k=1}^{n} \sum_{l=0}^{q_k-1} M_{k,l} \cdot f_N^{(l)}(\lambda_j) = \\
    = \sum_{k=1}^{n} \sum_{l=0}^{q_k-1} M_{k,l} \cdot \lim_{N\to\infty} f_N^{(l)}(\lambda_j)
    = \sum_{k=1}^{n} \sum_{l=0}^{q_k-1} M_{k,l} \cdot f^{(l)}(\lambda_j). \tag*{\qedhere}
  \end{multline*}
\end{proof}

































    \section{Twierdzenie o asymptotycznym zachowaniu $ \norm{\exp(At)} $}
      %!TEX root = ../RRZ.tex
%
\begin{definition}
  \emph{Wykładnikiem Lapunowa} macierzy $A$ nazywamy liczbę
  %
  \begin{equation*}
    \overline{\lambda} = \max \{ \real \lambda_k : k = 1, \ldots, n \}.
  \end{equation*}
\end{definition}
%
\begin{definition}
  \emph{Potęgą Lapunowa} macierzy $A$ nazywamy liczbę
  %
  \begin{equation*}
    \overline{l} = \max \bigl\{ l \geq 0 : \exists k \in \{1,\ldots,n\} \quad \real \lambda_k = \overline{\lambda} 
    \ \wedge \ M_{k,l} \not= 0 \bigr\}.
  \end{equation*}
\end{definition}
%
\begin{theorem}
  Dla każdej macierzy $A$ zachodzą nierówności:
%
  \begin{equation*}
    0 < \liminf_{t\to\infty} \frac{\norm*{e^{At}}}{t^{\overline{l}} e^{\overline{\lambda} t}} \leq
    \limsup_{t\to\infty} \frac{\norm*{e^{At}}}{t^{\overline{l}} e^{\overline{\lambda} t}} < \infty.
  \end{equation*}
\end{theorem}
%
\begin{proof}
  Łatwe.
\end{proof}

































    \section{Twierdzenie o minimach funkcji Lapunowa i~stabilności}
    
  \chapter{Zagadnienia}
    \section{Istnienie i jednoznaczność rozwiązań, rozwiązania wysycone (otwarte)}
      \begin{theorem}[Peano]
  Niech $y' = f(y,t)$, gdzie $y(t_0) = y_0$ oraz
  \[
    f \colon H = \overline{B} (y_0, b) \times [t_0 - a, t_0 + a] \longrightarrow \R^m.
  \]
  Załóżmy, że funkcja $f$ jest ciągła i oznaczmy
  \[
    M = \sup \big\{ \norm[\big]{f(y,t)} : (y, t) \in H \big\}.
  \]
  Wówczas dla $\alpha = \min(a, b/M)$ istnieje rozwiązanie $y(t)$ określone na
  przedziale $[t_0 - \alpha, t_0 + \alpha]$, spełniające warunek początkowy $y(t_0) = y_0$.
\end{theorem}

\begin{theorem}[Picard — Lindelöf]
  Niech $y' = f(y,t)$, gdzie $y(t_0) = y_0$ oraz
  \[
    f \colon H = \overline{B} (y_0, b) \times [t_0 - a, t_0 + a] \longrightarrow \R^m.
  \]
  Załóżmy, że funkcja $f$ jest ciągła oraz lipszycowska ze względu na $y$, to znaczy
  \[
    \exists L \ \forall (y_1,t), (y_2,t) \in H \quad
    \norm[\big]{f(y_1,t) - f(y_2,t)} \leq L \cdot \norm{y_1 - y_2}.
  \]
  Oznaczmy ponadto
  \[
    M = \sup \big\{ \norm[\big]{f(y,t)} : (y, t) \in H \big\}.
  \]
  Wówczas dla dowolnego $\alpha < \min(a, b/M, 1/L)$ istnieje dokładnie jedno rozwiązanie
  zagadnienia Cauchy'ego z warunkiem początkowym $y(t_0) = y_0$ określone na przedziale
  $[t_0 - \alpha, t_0 + \alpha]$.
\end{theorem}

\begin{proof}
  Jako ćwiczenie.
\end{proof}
    \section{Metoda Frobeniusa (metoda)}
      %!TEX root = ../RRZ.tex
%
Rozważmy regularne\footnote{$a_2(t) \not= 0$} równanie różniczkowe postaci
%
\begin{equation} \label{regrr1}
  a_2(t) y'' + a_1(t) y' + a_0(t) y = 0.
\end{equation}
%
\begin{definition}
  Punkt $t_0$ nazwiemy regularnie osobliwym, jeśli funkcja $\frac{a_2(t)}{(t-t_0)^2}$ jest analityczna w $t_0$ 
  i nie znika w $t_0$, a funkcja $\frac{a_1(t)}{t-t_0}$ jest analityczna w $t_0$.
\end{definition}
%
W przypadku punktu regularnie osobliwego równanie \eqref{regrr1} sprowadza się do
%
\begin{equation} \label{regrr2}
  (t-t_0)^2 y'' + (t-t_0) p(t) y' + q(t) y = 0.
\end{equation}
%
Rozwiązań będziemy szukali jedynie poza $t_0$, $t > t_0$.
%
\paragraph{Metoda Frobeniusa.} Niech $t_0 = 0$. Szukamy rozwiązań w postaci
%
\begin{equation*}
  y(t) = t^{\lambda} \sum_{n=0}^{\infty} c_n t^n,
\end{equation*}
%
gdzie $c_0 \not= 0$. Różniczkując stronami, dostajemy
%
\begin{equation*}
  t y'(t) = t^{\lambda} \sum_{n=0}^{\infty} c_n (n+\lambda) t^n, \qquad
  t^2 y''(t) = t^{\lambda} \sum_{n=0}^{\infty} c_n (n+\lambda) (n+\lambda-1) t^n.
\end{equation*}
%
Niech $p(t) = \sum_{n=0}^{\infty} p_n t^n$ oraz $q(t) = \sum_{n=0}^{\infty} q_n t^n$. Wtedy
%
\begin{align*}
  t p(t) y'(t) &= t^{\lambda} \p[\bigg]{\sum_{n=0}^{\infty} p_n t^n} \cdot \p[\bigg]{\sum_{n=0}^{\infty} c_n 
  (n+\lambda) t^n}
  = t^{\lambda} \sum_{n=0}^{\infty} t^n \sum_{k=0}^n c_k (k+\lambda) p_{n-k} = \\
  &= t^{\lambda} \sum_{n=0}^{\infty} t^n \p[\bigg]{c_n (n+\lambda) p_0 + \sum_{k=0}^{n-1} c_k (k+\lambda) p_{n-k}}, \\
  q(t) y(t) &= t^{\lambda} \p[\bigg]{\sum_{n=0}^{\infty} q_n t^n} \cdot \p[\bigg]{\sum_{n=0}^{\infty} c_n t^n}
  = t^{\lambda} \sum_{n=0}^{\infty} t^n \sum_{k=0}^n c_k q_{n-k} = \\
  &= t^{\lambda} \sum_{n=0}^{\infty} t^n \p[\bigg]{c_n q_0 + \sum_{k=0}^{n-1} c_k q_{n-k}}.
\end{align*}
%
Wstawiamy wynik do równania \eqref{regrr2}, otrzymując
%
\begin{multline*}
  t^{\lambda} \sum_{n=0}^{\infty} t^n \p[\big]{(n+\lambda) (n+\lambda-1) c_n + (n+\lambda) p_0 c_n + q_0 c_n } + \\
  + t^{\lambda} \sum_{n=0}^{\infty} t^n \underbrace{\p[\bigg]{\sum_{k=0}^{n-1} (k+\lambda) c_k p_{n-k} + 
    \sum_{k=0}^{n-1} c_k q_{n-k}}}_{-X_n(c_0,\ldots,c_{n-1})} = 0.
\end{multline*}
%
Dla każdej naturalnej liczby $n > 0$ zachodzi:
\begin{equation*}
  c_n \p[\big]{(n+\lambda) (n+\lambda-1) + (n+\lambda) p_0 + q_0} = X_n(c_0,\ldots,c_{n-1})
\end{equation*}
%
oraz $X_0 = 0$ dla $n = 0$. Zdefiniujmy \emph{wielomian indeksowy} $P$ wzorem
%
\begin{equation*}
  P(s) = s(s-1) + p_0 s + q_0.
\end{equation*}
%
Wtedy otrzymujemy
%
\begin{equation*}
  c_n P(n+\lambda) = X_n(c_0,\ldots,c_{n-1}).
\end{equation*}
%
Stąd wynika, że $c_0 P(\lambda) = 0$. Założyliśmy, że $c_0 \not= 0$, więc \emph{$\lambda$ musi być pierwiastkiem 
wielomianu indeksowego}. Dla $n = 0$ mamy $X_0 = 0$, więc dla $n > 0$ rekurencja przyjmuje postać
%
\begin{equation*}
  c_n = \frac{X_n(c_0,\ldots,c_{n-1})}{P(n+\lambda)}.
\end{equation*}
%
To pozwala wyliczyć współczynniki $c_n$ dla $n > 0$, chyba że $P(n+\lambda) = 0$ dla pewnego $n$, czyli $n+\lambda$ 
jest pierwiastkiem wielomianu indeksowego.
%
\paragraph{Przypadek podstawowy.} Wielomian indeksowy ma dwa pierwiastki rzeczywiste $\lambda_1,\lambda_2$ nieróżniące 
się o liczbę 
całkowitą. Wówczas otrzymujemy dwa liniowo niezależne rozwiązanie przyjmując $\lambda = \lambda_1, \lambda_2$.
%
\paragraph{Przypadek zespolony.} Bierzemy jeden z nich, dostając rozwiązanie zespolone
%
\begin{equation*}
  y(t) = t^{\lambda} \sum_{n=0}^{\infty} c_n t^n = e^{t \real \lambda} \p[\big]{\cos (t \imag \lambda) + i \sin 
  (t \imag \lambda)} \sum_{n=0}^{\infty} c_n t^n.
\end{equation*}
%
Rozwiązaniami są $\real y(t)$ oraz $\imag y(t)$, i są liniowo niezależne.
%
\paragraph{Przypadek pierwiastków różniących się o liczbę całkowitą.} Niech $\lambda$ oraz $\lambda +r$ będą 
pierwiastkami, gdzie $r \in \Z$. Jeśli $r=0$, to mamy pierwiastek podwójny. W przeciwnym przypadku otrzymujemy jedno 
rozwiązanie postaci
%
\begin{equation} \label{yzero}
  y_0(t) = t^{\lambda+r} \sum_{n=0}^{\infty} c_n t^n.
\end{equation}
%
Drugiego rozwiązania szukamy w postaci
%
\begin{equation*}
  y(t) = t^{\lambda} \sum_{n=0}^{\infty} d_n t^n + \gamma y_0(t) \ln t,
\end{equation*}
%
gdzie $\gamma$ to stała, którą wyznaczymy. Wstawiając do równania, otrzymujemy
%
\begin{multline*}
  t^{\lambda} \sum_{n=0}^{\infty} t^n \p[\big]{P(n+\lambda) \cdot d_n - X_n(d_0,\ldots,d_{n-1})} + \\ +
  \gamma t^2\p[\big]{y_0(t) \ln t}'' + \gamma t p(t) \p[\big]{y_0(t) \ln t}' + \gamma q(t) y_0(t) \ln t = 0.
\end{multline*}
%
Zajmijmy się składnikami zawierającymi $\gamma$. Po zróżniczkowaniu, dostajemy
%
\begin{align*}
  &\gamma \ln t \p[\big]{t^2 y_0''(t) + t p(t) y_0'(t) + q(t) y_0(t)}
    + \gamma \p[\big]{2t y_0'(t) + \p[\big]{p(t) - 1} y_0(t)} = \ldots
\intertext{Pierwszy człon się zeruje, bo $y_0$ jest rozwiązaniem. Wstawiając \eqref{yzero}, mamy}
  \ldots &= \gamma \p[\bigg]{2t \cdot \sum_{n=0}^{\infty} (n+\lambda+r) c_n t^{n+\lambda+r-1} + 
    \p[\big]{p(t)-1} \cdot \sum_{n=0}^{\infty} c_n t^{n+\lambda+r}} \\
  &= \gamma t^{\lambda+r} \p[\bigg]{\sum_{n=0}^{\infty} 2(n+\lambda+r) c_n t^n + p(t) \sum_{n=0}^{\infty} c_n t^n - 
  \sum_{n=0}^{\infty} c_n t^n} \\
  &= \gamma t^{\lambda+r} \p[\bigg]{\sum_{n=0}^{\infty} 2(n+\lambda+r) c_n t^n + 
    \sum_{n=0}^{\infty} t^n \sum_{k=0}^n c_k p_{n-k} - \sum_{n=0}^{\infty} c_n t^n} \\
  &= \gamma t^{\lambda+r} \sum_{n=0}^{\infty} t^n \p[\bigg]{2(n+\lambda+r) c_n + \sum_{k=0}^n c_k p_{n-k} - c_n} \\
  &= \gamma t^{\lambda+r} \sum_{n=0}^{\infty} t^n \p[\bigg]{2(n+\lambda+r) c_n + 
    (p_0-1) c_n + \sum_{k=0}^{n-1} c_k p_{n-k}} = \ldots \\
\intertext{Zauważmy, że $P'(n+\lambda+r) = 2(n+\lambda+r) + p_0 - 1$. Stąd}  
  \ldots &= \gamma t^{\lambda+r} \sum_{n=0}^{\infty} t^n \p[\big]{P'(n+\lambda+r) c_n + Y_n(c_0,\ldots,c_{n-1})} =\ldots
\intertext{gdzie $Y_n(c_0,\ldots,c_{n-1}) = \sum_{k=0}^{n-1} c_k p_{n-k}$. Przesuwając indeksy, dostajemy}
  \ldots &= \gamma t^{\lambda} \sum_{n=r}^{\infty} t^n \p[\big]{c_{n-r} P'(n+\lambda) + Y_{n-r}(c_0,\ldots,c_{n-r-1})}.
\end{align*}
%
Otrzymaliśmy zatem
%
\begin{multline*}
  \sum_{n=0}^{\infty} t^{n+\lambda} \p[\big]{ P(n+\lambda) d_n - X_n(d_0,\ldots,d_{n-1}) } \ + \\ +
  \gamma \sum_{n=r}^{\infty} t^{n+\lambda} \p[\big]{ P'(n+\lambda) c_{n-r} + Y_{n-r}(c_0,\ldots,c_{n-r}) } = 0.
\end{multline*}
%
Przyrównujemy do zera współczynniki przy $t^{n+\lambda}$. Jeśli $0 \leq n < r$, to
%
\begin{equation*}
  P(n+\lambda) d_n = X_n(d_0,\ldots,d_{n-1}).
\end{equation*}
%
Z kolei jeśli $n \geq r$, to mamy
%
\begin{equation*}
  P(n+\lambda) d_n + \gamma P'(n+\lambda) c_{n-r} = X_n(d_0,\ldots,d_{n-1}) + \gamma Y_{n-r}(c_0,\ldots,c_{n-r}).
\end{equation*}

\textsc{Przypadek 1.} Niech $r>0$. Wybieramy dowolne $d_0$, byle tylko różne od zera. Dla $n=1,\ldots,r-1$ otrzymujemy 
$d_n$ z rekurencji. Jeśli $n=r$, to
%
\begin{equation*}
  \gamma P'(\lambda+r) c_0 = X_r(d_0,\ldots,d_{r-1}).
\end{equation*}
%
Kładziemy
%
\begin{equation*}
  \gamma = \frac{X_r(d_0,\ldots,d_{r-1})}{c_0 P'(\lambda+r)}.
\end{equation*}
%
Powyższe wyrażenie ma sens, bo $\lambda+r$ nie jest pierwiastkiem podwójnym, więc $P'(\lambda+r) \not= 0$. Ponadto nie 
otrzymaliśmy warunku na $d_r$, więc $d_r$ może być dowolne (nawet $0$). W końcu, jeśli $n>r$, to
%
\begin{equation*}
  d_n = \frac{X_n - Y_{n-r} - \gamma P'(n+\lambda) c_{n-r}}{P(\lambda+r)}.
\end{equation*}

\textsc{Przypadek 2.} Niech $r=0$. Wtedy $P(\lambda) = P'(\lambda) = 0$. Kładziemy dowolne $d_0 \not= 0$ oraz $\gamma 
\not= 0$. Dla $n > 0$ kolejne współczynniki wyznaczamy tak samo, jak w przypadku poprzednim ($n>r$).

































    \section{Rozwiązania układów liniowych jednorodnych (otwarte)}
      %!TEX root = ../RRZ.tex
%
\begin{definition}
  Równaniem liniowym nazywamy równanie postaci
  %
  \begin{equation*}
    \frac{dy}{dt} = A(t)y + B(t),
  \end{equation*}
  %
  gdzie $A(t)$ jest macierzą $m \times m$, a $B(t)$ wektorem z $\R^m$ o ciągłych współczynnikach, określonym na 
  przedziale otwartym $I \subset \R$.
\end{definition}
%
\begin{definition}
  Równanie $y' = A(t)y$ nazywamy \emph{jednorodnym}, a równanie $y' = A(t)y + B(t)$ (odpowiadającym) niejednorodnym.
\end{definition}

\begin{theorem}
  Zbiór rozwiązań równania jednorodnego jest podprzestrzenią liniową $C^0(I,\R^m)$, a zbiór rozwiązań równania 
  niejednorodnego jej warstwą.
\end{theorem}

\noindent Niech $V$ oznacza zbiór rozwiązań wysyconych równania $y' = A(t)y$.

%\begin{statement}
%  Następujące warunki są równoważne dla zbioru $\{y_1, \ldots, y_n\} \subset V$:
%  \begin{enumerate}
%    \item Zbiór jest liniowo niezależny.
%    \item Dla dowolnego $t \in I$ zbiór $\big\{y_1(t), \ldots, y_n(t)\big\}$ jest liniowo niezależny w $\R^m$.
%    \item Istnieje $t \in I$, że zbiór $\big\{y_1(t), \ldots, y_n(t)\big\}$ jest liniowo niezależny w $\R^m$.
%  \end{enumerate}
%\end{statement}
    \section{Rozwiązywanie równań liniowych niejednorodnych}
    \section{Hiperboliczność i stabilność punktów równowagi}
      \input{subfiles/2-5.tex}
    \section{Zagadnienia brzegowe}
      \input{subfiles/2-6.tex}
  
  \chapter{Przykłady}
    \section{Rozwiązywanie równań metodą szeregów potęgowych}
    \section{Równania na wariację}
    \section{Potoki - policzenie i zastosowanie własności w konkretnych sytuacjach}
    \section{Wzory Liouville'a i Abela}
    \section{Zastosowania twierdzenia spektralnego, macierze spektralne}
    \section{Całki pierwsze, funkcje Lapunowa - zastosowanie do badania stabilności}
\end{document}
































