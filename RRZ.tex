\documentclass{mwart}

\usepackage[utf8]{inputenc}
\usepackage[T1, nomathsymbols]{polski}
\usepackage[boldsans]{concmath}
\usepackage{concrete}
\usepackage{amsmath}
\usepackage{amsthm}
\usepackage{amssymb}
\usepackage{amsfonts}
\usepackage{mathtools}
\usepackage{thmtools}
\usepackage[paperheight=297mm, paperwidth=210mm, top=2.5cm, bottom=2.5cm, left=4.4cm, right=4.4cm]{geometry}

\declaretheoremstyle
[
  spaceabove=\topsep,
  spacebelow=\topsep,
  headfont=\bfseries,
  notefont=\bfseries,
  notebraces={(}{)},
  bodyfont=\normalfont,
]
{theorem}

\declaretheoremstyle
[
  spaceabove=\topsep,
  spacebelow=\topsep,
  headfont=\itshape \bfseries,
  notefont=\normalfont,
  notebraces={(}{)},
  bodyfont=\normalfont,
  qed=\ensuremath{\blacksquare},
]
{proof}
\let\proof\relax

\declaretheorem[style=theorem, name=Twierdzenie, numberwithin=subsection]{theorem}
\declaretheorem[style=theorem, name=Lemat,       numberlike=theorem]{lemma}
\declaretheorem[style=theorem, name=Definicja,   numberlike=theorem]{definition}
\declaretheorem[style=theorem, name=Wniosek,     numberlike=theorem]{conclusion}
\declaretheorem[style=theorem, name=Przykład,    numberlike=theorem]{example}
\declaretheorem[style=proof,   name=Dowód,       numbered=no]{proof}

\newcommand*{\R}{\mathbb{R}}
\newcommand*{\N}{\mathbb{N}}
\newcommand*{\Q}{\mathbb{Q}}
\newcommand*{\Z}{\mathbb{Z}}
\newcommand*{\C}{\mathbb{C}}

\DeclarePairedDelimiter\abs{\lvert}{\rvert}
\DeclarePairedDelimiter\norm{\lVert}{\rVert}
\DeclarePairedDelimiter\p{(}{)}

\pagestyle{myuheadings}

\title{
  \Huge {\bfseries Równania różniczkowe zwyczajne} \\
  \Large {\scshape Opracowanie zagadnień na egzamin}
}
\author{{\scshape KJG}}
\date{{\scshape Wersja z 15 grudnia 2018 r.}}

\begin{document}
	
  \maketitle
  \newpage
  
  \tableofcontents
  \newpage
  
  \section{Twierdzenia}
    \subsection{Ciągła zależność od parametru}
    \subsection{Różniczkowalna zależność od parametrów}
    \subsection{Rozwiązania przez szeregi potęgowe wokół punktu regularnego}
    \subsection{Twierdzenie spektralne dla funkcji analitycznych}
    \subsection{Twierdzenie o asymptotycznym zachowaniu $\mathbf{\left\lVert e^{At} \right\rVert}$}
    \subsection{Twierdzenie o minimach funkcji Lapunowa i stabilności}
  
  \section{Zagadnienia}
    \subsection{Istnienie i jednoznaczność rozwiązań, rozwiązania wysycone}
      \begin{theorem}
  Niech $y' = f(y,t)$ oraz $y(t_0) = y_0$. Załóżmy, że funkcja $f$ jest ciągła na zbiorze
  $H = \overline{B}(y_0, b) \times [t_0 - a, t_0 + a]$, gdzie $a, b > 0$. Oznaczmy
  \[
    M = \sup \big\{ \norm[\big]{f(y,t)} : (y, t) \in H \big\}.
  \]
\end{theorem}
    \subsection{Metoda Frobeniusa}
    \subsection{Rozwiązania układów liniowych jednorodnych}
    \subsection{Rozwiązywanie równań liniowych niejednorodnych}
    \subsection{Hiperboliczność i stabilność punktów równowagi}
    \subsection{Zagadnienia brzegowe}
  
  \section{Przykłady}
    \subsection{Rozwiązywanie równań metodą szeregów potęgowych}
    \subsection{Równania na wariację}
    \subsection{Potoki -- policzenie i zastosowanie własności w konkretnych sytuacjach}
    \subsection{Wzory Liouville'a i Abela}
    \subsection{Zastosowania twierdzenia spektralnego, macierze spektralne}
    \subsection{Całki pierwsze, funkcje Lapunowa -- zastosowanie do badania stabilności}

\end{document}













