%!TEX root = ../RRZ.tex

\documentclass{mwrep}

\usepackage[utf8]{inputenc}
\usepackage[T1, nomathsymbols]{polski}
\usepackage{lmodern}
\usepackage{amsmath, amsthm, amssymb, amsfonts}
\usepackage{mathtools}
\usepackage{thmtools}
\usepackage[
  paperheight = 297mm,
  paperwidth  = 210mm,
  top         = 2.5cm,
  bottom      = 2.5cm,
  left        = 4.4cm,
  right       = 4.4cm
]{geometry}

\declaretheoremstyle[
  spaceabove = \topsep,
  spacebelow = \topsep,
  headfont   = \bfseries,
  notefont   = \bfseries,
  notebraces = {(}{)},
  bodyfont   = \normalfont,
]{mystyle}

\let\proof\relax

\declaretheorem[
  style  = mystyle,
  name   = Twierdzenie,
  parent = section
]{theorem}
\declaretheorem[
  style  = mystyle,
  name   = Stwierdzenie,
  sibling = theorem
]{statement}
\declaretheorem[
  style   = mystyle,
  name    = Lemat,
  sibling = theorem
]{lemma}
\declaretheorem[
  style   = mystyle,
  name    = Definicja,
  sibling = theorem
]{definition}
\declaretheorem[
  style   = mystyle,
  name    = Wniosek,
  sibling = theorem
]{conclusion}
\declaretheorem[
  style   = mystyle,
  name    = Przykład,
  sibling = theorem
]{example}
\declaretheorem[
  style    = mystyle,
  name     = Dowód,
  numbered = no,
  qed      = \ensuremath{\blacksquare}
]{proof}

\newcommand*{\R}{\mathbb{R}}
\newcommand*{\N}{\mathbb{N}}
\newcommand*{\Q}{\mathbb{Q}}
\newcommand*{\Z}{\mathbb{Z}}
\newcommand*{\C}{\mathbb{C}}

\DeclarePairedDelimiter\abs{\lvert}{\rvert}
\DeclarePairedDelimiter\norm{\lVert}{\rVert}
\DeclarePairedDelimiter\p{(}{)}

\DeclareMathOperator{\Dm}{Dm}
\DeclareMathOperator{\spectre}{sp}

\newcommand{\goodchi}{\protect\raisebox{2pt}{$\chi$}}

\pagestyle{uheadings}
\makeatletter
\renewcommand\heading@font\scshape
\makeatother

\title{
  \huge \textbf{Równania różniczkowe zwyczajne} \\
  \Large Opracowanie zagadnień na egzamin
}
\author{ KJG }
\date{ Wersja z \today }

\begin{document}

  \maketitle
  \newpage~
  \thispagestyle{empty}
  \tableofcontents
  
  \chapter{Twierdzenia}
    \section{Ciągła zależność od parametrów}
      %!TEX root = ../RRZ.tex

\begin{theorem}[o ciągłej zależności od parametru]
  Niech
  \[
    y' = f(y,t,\lambda), \qquad
    f \colon \R^{m + 1} \times \R^l \supset U \times B_l(\lambda_0,c) \longrightarrow \R^m,
  \]
  gdzie $f$ jest funkcją ciągłą oraz $c > 0$. Niech $y(t,\lambda_0)$ będzie rozwiązaniem równania
  $y' = f(y,t,\lambda_0)$ z warunkiem początkowym $(y_0, t_0)$ określonym na \emph{zwartym}
  przedziale $I$ zawierającym $t_0$. Wybierzmy $b > 0$ i rozważmy zbiór 
  \[
    R_b = \Big\{ (y,t) : t \in I, \norm[\big]{ y - y(t,\lambda_0) } < b \Big\}.
  \]
  Załóżmy dalej, że
  \begin{enumerate}
    \item $\exists L \geq 0 \ \forall (y_1,t),(y_2,t) \in R_b \quad
    \norm[\big]{ f(y_1,t,\lambda_0) - f(y_2,t,\lambda_0) } \leq L \cdot \norm{y_1 - y_2}$,
    \item $\forall \varepsilon > 0 \ \exists \delta > 0 \ \forall (y,t) \in R_b
    \ \forall \lambda \colon \norm{\lambda - \lambda_0} < \delta \quad \norm[\big]{ f(y,t,\lambda) 
    - f(y,t,\lambda_0) } < \varepsilon$.
  \end{enumerate}
  Wówczas istnieje stała $c^{\ast} > 0$ taka, że
  \begin{enumerate}
    \item jeśli $\norm{\lambda - \lambda_0} < c^{\ast}$, to $y(t,\lambda)$ jest określone na $I$
    \item jeśli $\lambda_n \to \lambda_0$, to $y(t,\lambda_n) \rightrightarrows y(t,\lambda_0)$ na $I$.
  \end{enumerate}
\end{theorem}
    \section{Różniczkowalna zależność od parametrów}
      %!TEX root = ../RRZ.tex

\begin{theorem}[o różniczkowalnej zależności od parametru]
  Niech
  \[
    y' = f(y,t,\lambda), \qquad 
    f \colon \R^{m+1} \times \R \supset U \times (\lambda_0 - c, \lambda_0 + c) \longrightarrow \R^m,
  \]
  gdzie $f$ jest funkcją ciągłą względem $y,t,\lambda$ oraz klasy $C^1$ względem $y,\lambda$. Ustalmy
  warunek początkowy $(y_0,t_0)$ i oznaczmy przez $y(t,\lambda)$ rozwiązanie równania
  \[
    \frac{\partial y(t,\lambda)}{\partial t} = f(y,t,\lambda)
  \]
  z warunkiem początkowym $y(t_0,\lambda) = y_0$, określone na ustalonym i \emph{zwartym} przedziale $I$.
  Wówczas na przedziale $I$ istnieje ciągła funkcja
  \[
    z(t,\lambda_0) = \frac{\partial y(t,\lambda)}{\partial \lambda} \Big\vert_{\lambda = \lambda_0}
  \]
  oraz zachodzi równość
  \[
    \frac{\partial z(t,\lambda_0)}{\partial t} =
    \frac{\partial^2 y(t,\lambda)}{\partial t \partial \lambda} \Big\vert_{\lambda = \lambda_0} =
    \frac{\partial^2 y(t,\lambda)}{\partial \lambda \partial t} \Big\vert_{\lambda = \lambda_0}.
  \]
\end{theorem}
    \section{Rozwiązania przez szeregi potęgowe wokół punktu regularnego}
      %!TEX root = ../RRZ.tex
%
Rozważmy równanie
%
\begin{equation} \label{secondorder1}
  a_2(t) y'' + a_1(t)y' + a_0(t)y = 0,
\end{equation}
%
gdzie $a_2,a_1,a_0$ są analityczne w pewnym punkcie $t_0$.
%
\begin{definition}
  Powiemy, że $t_0$ jest \emph{punktem regularnym} wtedy i tylko wtedy, gdy $a_2(t_0) \not= 0$. W przeciwnym wypadku 
  $t_0$ nazwiemy \emph{punktem osobliwym}.
\end{definition}
%
W przypadku regularnym równanie \eqref{secondorder1} sprowadza się do
%
\begin{equation} \label{secondorder2}
  y'' + p(t)y' + q(t)y = 0,
\end{equation}
%
gdzie $p$ i $q$ są analityczne w punkcie $t_0$, czyli
%
\begin{equation*}
  p(t) = \sum_{n=0}^{\infty} p_n(t-t_0)^n, \qquad q(t) = \sum_{n=0}^{\infty} q_n(t-t_0)^n.
\end{equation*}
%
\begin{theorem}
  Każde rozwiązanie równania \eqref{secondorder2} jest analityczne w kole, w którym oba szeregi $p(t)$ i $q(t)$ 
  zbiegają. Co więcej, analityczna funkcja
%
  \begin{equation*}
    y(t) = \sum_{n=0}^{\infty}c_n(t - t_0)^n
  \end{equation*}
%
  jest rozwiązaniem wtedy i tylko wtedy, gdy
%
  \begin{equation} \label{rps}
    c_{n+2} = - \frac{1}{(n+1)(n+2)} \p*{ \sum_{k=0}^{n} c_{k+1}(k+1)p_{n-k} + \sum_{k=0}^{n} c_k q_{n-k} }.
  \end{equation}
%
\end{theorem}
%
\begin{proof}
  Dla ustalenia uwagi niech $t_0 = 0$ oraz $y(t) = \sum_{n=0}^{\infty} c_nt^n$. Wtedy
%
  \begin{equation*}
    y'(t) = \sum_{n=0}^{\infty} (n+1) c_{n+1} t^n, \qquad y''(t) = \sum_{n=0}^{\infty} (n+1)(n+2) c_{n+2} t^n.
  \end{equation*}
%
  Z iloczynu Cauchy'ego%
  \footnote{
    $ \p*{ \sum_{n=0}^{\infty} a_n } \cdot \p*{ \sum_{n=0}^{\infty} b_n } =
    \sum_{n=0}^{\infty} \sum_{k=0}^{n} a_k b_{n-k} $
  }
  dostajemy
%
  \begin{flalign*}
    p(t) y'(t) &= \p*{ \sum_{n=0}^{\infty} p_n t^n} \cdot \p*{\sum_{n=0}^{\infty} (n+1) c_{n+1} t^n} 
        = \sum_{n=0}^{\infty} t^n \sum_{k=0}^{n} (k+1) c_{k+1} p_{n-k}, & \\
    q(t) y(t) &= \p*{ \sum_{n=0}^{\infty} q_n t^n } \cdot \p*{ \sum_{n=0}^{\infty} c_n t^n } 
        = \sum_{n=0}^{\infty} t^n \sum_{k=0}^{n} c_k q_{n-k}.
  \end{flalign*}
%
  Rozpisując lewą stronę równania \eqref{secondorder2}, otrzymujemy
%
  \begin{equation*}
    \sum_{n=0}^{\infty} t^n \p*{ (n+1)(n+2)c_{n+2} + \sum_{k=0}^{n} (k+1) c_{k+1} p_{n-k} + 
    \sum_{k=0}^{n} c_k q_{n-k} } = 0.
  \end{equation*}
%
  Z analityczności, dla każdego $n \geq 0$ jest
%
  \begin{equation*}
    (n+1)(n+2)c_{n+2} + \sum_{k=0}^{n} (k+1) c_{k+1} p_{n-k} + \sum_{k=0}^{n} c_k q_{n-k} = 0,
  \end{equation*}
%
  co dowodzi wzoru \eqref{rps}.
  
  Wzór rekurencyjny \eqref{rps} zadaje współczynniki $c_n$ dla $n \geq 2$, jeśli wybrane zostały $c_0$, $c_1$. 
  Zauważmy, że $c_0=y(t_0)$, $c_1=y'(t_0)$. Zatem dobierając $c_0$ oraz $c_1$ możemy otrzymać dowolny warunek 
  początkowy dla $y$, co pozwala uzyskać każde rozwiązanie wysycone. Pozostaje pokazać, że przy dowolnym wyborze $c_0$, 
  $c_1$ wzór \eqref{rps} prowadzi do szeregu Taylora funkcji analitycznej w kole $D(t_0,R)$.
  
  Wybierzmy $0 < r < R$. Wtedy funkcje $p,q$ są zbieżne bezwzględnie w $\overline{D} (t_0,r)$, więc istnieją stałe 
  $L_p$ oraz $L_q$ takie, że dla dowolnego $n \geq 0$ jest
%
  \begin{equation*}
    \abs{p_n} r^n \leq L_p, \qquad \abs{q_n} r^n \leq L_q.
  \end{equation*}
%
  Niech $0 < \rho < r$ oraz $\gamma_n = \abs{c_n} \rho^n$, $\Gamma_n = \max\{\gamma_j: j = 1,\ldots,n\}$. Wtedy
%
  \begin{align*}
    \abs{\gamma_{n+2}}
    &\leq \frac{\rho^{n+2}}{(n+1)(n+2)} \p*{ \sum_{k = 0}^{n} (k + 1) \cdot \abs{c_{k + 1}} \cdot \abs{p_{n - k}} + 
        \sum_{k = 0}^{n} \abs{c_k} \cdot \abs{q_{n - k}}} \\ 
    &\leq \frac{\rho^{n+2}}{(n+1)(n+2)} \p*{ \sum_{k=0}^{n} (n+1) \cdot \frac{\gamma_{k+1}}{\rho^{k+1}} 
        \cdot \frac{L_p}{r^{n-k}} + \sum_{k=0}^{n} \frac{\gamma_k}{\rho^k} \cdot \frac{L_q}{r^{n-k}} } \\ 
    &\leq \frac{\rho^{n+2}}{(n+1)(n+2)} \p*{ (n+1) \sum_{k=0}^{n} \frac{\Gamma_{n+1}}{\rho^{k-n} \rho^{n+1}} \cdot 
        \frac{L_p}{r^{n-k}} + \sum_{k=0}^{n} \frac{\Gamma_n}{\rho^{k-n}\rho^{n}} \cdot \frac{L_q}{r^{n-k}} } \\
    &\leq \frac{\rho L_p}{n+2} \Gamma_{n+1} \sum_{k=0}^{n} \p*{ \frac{\rho}{r} }^{n-k} +
        \frac{\rho^2 L_q}{(n+1)(n+2)} \Gamma_n \sum_{k=0}^{n} \p*{ \frac{\rho}{r} }^{n-k} \\
    &\leq \p*{ \frac{\rho L_p}{n+2} + \frac{\rho^2 L_q}{(n+2)(n+1)}} \Gamma_{n+1}
        \sum_{k=0}^n \p*{ \frac{\rho}{r} }^{n-k} \\
    &\leq \underbrace{\p*{ \frac{\rho L_p}{n+2} + \frac{\rho^2 L_q}{(n+2)(n+1)}} \cdot
        \p*{\frac{1}{1 - \frac{\rho}{r}}}}_{\alpha_n} \Gamma_{n+1}.
  \end{align*}
%
  Wyrażenie $\alpha_n$ zbiega do zera, gdy $n \to \infty$. Istnieje $n_0$, że $\alpha_n < 1$ dla $n \geq n_0$. Zatem 
  dla każdego $n \geq n_0$ zachodzi $\gamma_{n+2} \leq \Gamma_{n+1}$, co jest równoważne temu, że
  $\Gamma_{n+2} = \Gamma_{n+1}$, czyli ciąg $\Gamma_n$ jest stały od pewnego miejsca i ograniczony przez pewne 
  $\overline{\Gamma}$. Stąd, jeśli $\abs{t} < \rho$, to z kryterium Cauchy'ego jest
%
  \begin{equation*}
    \sqrt[n]{\abs{c_n} \cdot \abs{t}^n} = \sqrt[n]{\abs{c_n} \cdot \rho^n} \cdot 
    \sqrt[n]{\frac{\abs{t}^n}{\rho^n}} \leq \sqrt[n]{\overline{\Gamma}} \cdot \abs*{\frac{t}{\rho}} < 1,
  \end{equation*}
%
  o ile $n$ jest dostatecznie duże. Wobec tego szereg $\sum_{n=0}^{\infty} c_n t^n$ jest zbieżny w kole o promieniu 
  $\rho$. Ponieważ $\rho$ może być dowolnie bliskie $R$, to suma kół wypełnia koło otwarte o~promieniu $R$, co kończy 
  dowód.
\end{proof}
































\pagebreak
    \section{Twierdzenie spektralne dla funkcji analitycznych}
      %!TEX root = ../RRZ.tex
%
\begin{definition}
  \emph{Widmem} macierzy $A$ nazywamy zbiór jej wartości własnych wraz z krotnościami, i oznaczamy $\spectrum (A)$.
\end{definition}
%
\begin{theorem}[Hamilton-Cayley] \label{T: H-C}
  Dla każdej macierzy $A$ zachodzi $\mychi_A(A) = 0$, gdzie $\mychi_A$ jest wielomianem charakterystycznym macierzy $A$.
\end{theorem}
%
\begin{theorem}[Spektralne dla wielomianów]
  Niech $A$ będzie macierzą o wartościach własnych $\lambda_1,\ldots,\lambda_n$ z krotnościami $q_1,\ldots,q_n$. Wtedy 
  istnieją macierze $M_{k,l}$ dla $1 \leq k \leq n$, $0 \leq l \leq q_k-1$, zwane \emph{spektralnymi} takie, że dla 
  każdego wielomianu $f$ zachodzi:
  \begin{equation*}
    f(A) = \sum_{k=1}^{n} \sum_{l=0}^{q_k-1} M_{k,l} \cdot f^{(l)}(\lambda_j).
  \end{equation*}
\end{theorem}
%
\begin{proof}
  W celu udowodnienia twierdzenia będzie potrzebny lemat pomocniczy.
  %
  \begin{nestedlemma} \label{L: zetdoer}
    Macierze $M_{k,l}$ są jednoznacznie wyznaczone przez tezę twierdzenia spektralnego dla
    $f(z) = z^r$, $r = 0, \ldots, n-1$.
  \end{nestedlemma}
  %
  \begin{nestedproof}
    Otrzymujemy układ równań z niewiadomymi $M_{k,l}$, czyli
    %   
    \begin{equation*}
      A^r = \sum_{k=1}^n \sum_{l=0}^{q_k-1} M_{k,l} \cdot r (r-1) \cdots (r-l+1) \cdot \lambda_k^{r-l} c_r = 0.
    \end{equation*}
    %   
    Teza lematu oznacza, że układ ten jest oznaczony. Pokażemy liniową niezależność wierszy. Wybierzmy współczynniki 
    $c_r$ dla $r = 1, \ldots, n-1$, tak aby kombinacja liniowa wierszy z tymi współczynnikami wynosiła $0$, czyli dla
    każdych $k = 1,\ldots,n$ oraz $l = 0, \ldots, q_k-1$ jest
    %   
    \begin{equation*}
      \sum_{r=0}^{n-1} c_r \cdot r (r-1) \cdots (r-l+1) \lambda_k^{r-l} = 0.
    \end{equation*}
    %
    Rozważmy wielomian $w(z) = \sum_{r=0}^{n-1} c_r z^r$. Otrzymaliśmy, że $w^{(l)}(\lambda_k) = 0$, czyli 
    $\lambda_k$ jest zerem z krotnością co najmniej $q_k$, a zatem suma krotności zer wielomianu $w$ jest równa co 
    najmniej $\sum_{k=1}^n q_k = n$, co jest sprzecznością, bo stopień wielomianu był co najwyżej $n-1$.
  \end{nestedproof}
  %
  Twierdzenie zostanie udowodnione indukcyjnie ze względu na stopień $f$.
  
  Przypuśćmy, że twierdzenie zachodzi dla wielomianów stopnia mniejszego od $n+r$, gdzie $r \geq 0$. Z lematu 
  \ref{L: zetdoer} teza zachodzi dla $r=0$. Zwróćmy uwagę, że obie strony twierdzenia są liniowe względem $f$. 
  Wystarczy więc pokazać je dla układu rozpinającego przestrzeń wielomianów stopnia mniejszego niż $n+r$. W~celu 
  pokazania, że twierdzenia zachodzi również dla wielomianów stopnia $n+r$, wystarczy pokazać dla $f_r(z) = z^r 
  \mychi_A(z)$, bo każdy wielomian
  %
  \begin{equation*}
    f(z) = a_{n+r} z^{n+r} + \ldots + a_1 z + a_0
  \end{equation*}
  %
  można zapisać jako
  %
  \begin{equation*}
    f(z) = f_r(z) + P(z),
  \end{equation*}
  %
  gdzie $P$ jest wielomianem stopnia mniejszego niż $n+r$. Zauważmy, że
  %
  \begin{equation*}
    L = f_r(A) = A^r \cdot \mychi_A(A) \overset{\ref{T: H-C}}{=} 0, \qquad 
    P = \sum_{k=1}^n \sum_{l=0}^{q_k-1} M_{k,l} \cdot f^{(l)}(\lambda_k),
  \end{equation*}
  %
  a ponadto $f_r(z) = (z-\lambda_k)^{q_k} \cdot Q(z)$. Pochodne rzędu niższego od $q_k$ składają się z sum członów, w 
  których $(z-\lambda_k)$ występuje w dowolnej potędze, więc zerują się przy podstawieniu $z=\lambda_k$.
  %
\end{proof}
%
\begin{theorem}[Spektralne dla funkcji analitycznych]
  Niech
  %
  \begin{equation*}
    f(z) = \sum_{n=0}^{\infty} a_n z^n, \quad \abs{z} < R.
  \end{equation*}
  %
  Załóżmy, że $\spectrum(A) \subset D(0,R)$. Wówczas szereg $f(A)$ zbiega i zachodzi teza twierdzenia spektralnego dla 
  wielomianów:
  %
  \begin{equation*}
    f(A) = \sum_{k=1}^{n} \sum_{l=0}^{q_k-1} M_{k,l} \cdot f^{(l)}(\lambda_j).
  \end{equation*}
\end{theorem}
%
\begin{proof}
  Oznaczmy
  %
  \begin{equation*}
    f_N(z) = \sum_{n=0}^N a_n z^n.
  \end{equation*}
  Korzystając z twierdzenia spektralnego dla wielomianów, dostajemy
  \begin{multline*}
    f(A) = \lim_{N\to\infty} f_N(A) 
    = \lim_{N\to\infty} \sum_{k=1}^{n} \sum_{l=0}^{q_k-1} M_{k,l} \cdot f_N^{(l)}(\lambda_j) = \\
    = \sum_{k=1}^{n} \sum_{l=0}^{q_k-1} M_{k,l} \cdot \lim_{N\to\infty} f_N^{(l)}(\lambda_j)
    = \sum_{k=1}^{n} \sum_{l=0}^{q_k-1} M_{k,l} \cdot f^{(l)}(\lambda_j). \tag*{\qedhere}
  \end{multline*}
\end{proof}

































    \section{Twierdzenie o asymptotycznym zachowaniu $ \norm{e^{At}} $}
      %!TEX root = ../RRZ.tex
%
\begin{definition}
  \emph{Wykładnikiem Lapunowa} macierzy $A$ nazywamy liczbę
  %
  \begin{equation*}
    \overline{\lambda} = \max \{ \real \lambda_k : k = 1, \ldots, n \}.
  \end{equation*}
\end{definition}
%
\begin{definition}
  \emph{Potęgą Lapunowa} macierzy $A$ nazywamy liczbę
  %
  \begin{equation*}
    \overline{l} = \max \bigl\{ l \geq 0 : \exists k \in \{1,\ldots,n\} \quad \real \lambda_k = \overline{\lambda} 
    \ \wedge \ M_{k,l} \not= 0 \bigr\}.
  \end{equation*}
\end{definition}
%
\begin{lemma}[Wzór Leibniza]
  \begin{equation} \label{Leibniz}
    D^n(fg) = \sum_{k=0}^n \binom{n}{k} D^k f \cdot D^{n-k} g.
  \end{equation}
\end{lemma}
%
\begin{theorem}
  Dla każdej macierzy $A$ zachodzą nierówności:
%
  \begin{equation*}
    0 < \liminf_{t\to\infty} \frac{ \norm{e^{At}} }{ t^{\overline{l}} e^{\overline{\lambda} t} } \leq
    \limsup_{t\to\infty} \frac{ \norm{e^{At}} }{ t^{\overline{l}} e^{\overline{\lambda} t} } < \infty.
  \end{equation*}
\end{theorem}
%
\begin{proof}
  Nierówność środkowa jest oczywista. Zaczniemy wobec tego od prawej.
%
  \begin{align*}
    \limsup_{t\to\infty} \frac{ \norm{e^{At}} }{ t^{\overline{l}} e^{\overline{\lambda} t}} &=
    \limsup_{t\to\infty} \frac{\norm*{\sum\limits_{k=1}^n \sum\limits_{l=0}^{q_k - 1} M_{k,l} \cdot t^l e^{\lambda_k 
    t}}}{t^{\overline{l}} \exp(\overline{\lambda}t)} \\ &\leq
    \sum_{k=1}^n \sum_{l=0}^{q_k-1} \norm{M_{k,l}} \cdot \limsup_{t\to\infty} \frac{t^l e^{(\real \lambda_k) 
    t}}{t^{\overline{l}} e^{\overline{\lambda}t}} \\ &=
    \sum_{k=1}^n \sum_{l=0}^{q_k-1} \norm{M_{k,l}} \cdot \limsup_{t\to\infty} t^{l-\overline{l}} e^{(\real \lambda_k - 
    \overline{\lambda}) t}
  \end{align*}
%
  Zauważmy, że $\real \lambda_k - \overline{\lambda} \leq 0$, a jeśli $\real \lambda_k - \overline{\lambda} = 0$ oraz 
  $M_{k,l} \not= 0$, to $l \leq \overline{l}$. Wobec tego, dla każdej kombinacji $k$ i $l$ jest
%
  \begin{equation*}
    \limsup_{t\to\infty} t^{l-\overline{l}} \exp \p[\big]{(\real \lambda_k - \overline{\lambda}) t} \leq 1.
  \end{equation*}
%
  Bez utraty ogólności możemy przyjąć $\overline{\lambda} = \lambda_1$, $M_{1,\overline{l}} \not= 0$ oraz
%
  \begin{equation*}
    w(z) = \prod_{k=2}^{n} (z-\lambda_k)^{q_k} = \frac{\mychi_A(z)}{(z-\lambda_1)^{q_1}}.
  \end{equation*}
%
  Niech $f(z) = w(z) \cdot e^{zt}$. Wtedy dla $k>0$ oraz $l<q_k$ jest $f^{(l)}(\lambda_k) = 0$, bo pochodna jest sumą 
  członów ze wzoru \eqref{Leibniz}, gdzie $(z-\lambda_k)$ występuje w~potędze dodatniej. Niechaj teraz $k=1$ oraz 
  $\varphi(z) = e^{zt}$, $\psi(z) = w(z)$. Wtedy
%
  \begin{equation*}
    (\varphi \cdot \psi)^{(l)} (\lambda_1) = \sum_{i=0}^l \binom li t^i e^{\lambda_1 t} w^{(l-i)}(\lambda_i) =
    e^{\lambda_1 t} \cdot p_l(t),
  \end{equation*}
%
  gdzie $p_l$ jest wielomianem stopnia co najwyżej $l$. Wtedy $p_{\overline{l}}$ ma postać
%
  \begin{equation*}
    t^{\overline{l}} e^{\lambda_1 t} w(\lambda_1) + \widetilde{p}_{\overline{l}}(t) e^{\lambda_1 t},
  \end{equation*}
%
  gdzie stopień $\widetilde{p}_{\overline{l}}$ jest mniejszy od $\overline{l}$. Z twierdzenia spektralnego jest
%
  \begin{align*}
    \liminf_{t\to\infty} \frac{\norm[\big]{w(A) e^{At}}}{t^{\overline{l}} e^{\lambda_1 t}} &=
    \liminf_{t\to\infty} \frac{\norm[\bigg]{\sum\limits_{l=0}^{\overline{l}-1} M_{1,l} e^{\lambda_1 t} 
    p_l(t) + 
    M_{1,\overline{l}} e^{\lambda_1 t} \p[\big]{t^{\overline{l}} w(\lambda_1) + \widetilde{p}_{\overline{l}} (t)} }} 
    {t^{\overline{l}} e^{\lambda_1 t}} \\
    &= \liminf_{t\to\infty} \frac{\norm[\big]{e^{\lambda_1 t}} \cdot 
    \norm[\bigg]{\sum\limits_{l=0}^{\overline{l}-1} 
    M_{1,l} 
    p_l(t) + M_{1,\overline{l}} \p[\big]{t^{\overline{l}} w(\lambda_1) + \widetilde{p}_{\overline{l}} (t)} }} 
    {t^{\overline{l}} e^{\lambda_1 t}} \\
    &\geq \liminf_{t\to\infty} \frac{ \norm{M_{1,\overline{l}}} \cdot \abs[\big]{ t^{\overline{l}} w(\lambda_1) + 
    \widetilde{p}_{\overline{l}}(t) } - \sum\limits_{l=0}^{\overline{l}-1} \norm{M_{1,\overline{l}}} \cdot 
    \norm{p_l(t)} }{t^{\overline{l}}} \\
    &\geq \liminf_{t\to\infty} \norm{M_{1,\overline{l}}} \cdot \abs*{w(\lambda_1) + 
    \frac{\widetilde{p}_{\overline{l}}(t)}{t^{\overline{l}}}} - \limsup_{t\to\infty} \sum_{l=0}^{\overline{l}-1} 
    \norm{M_{1,l}} \cdot \abs*{\frac{p_l(t)}{t^{\overline{l}}}} \\ 
    &= \norm{M_{1,\overline{l}}} \cdot 
    \underbrace{\abs[\big]{w(\lambda_1)}}_{>0}.
  \end{align*}
%
  Ostatecznie
%
  \begin{equation*}
    \liminf_{t\to\infty} \frac{\norm[\big]{w(A) e^{At}}}{t^{\overline{l}} e^{\lambda_1 t}} \geq 
    \frac{1}{\norm[\big]{w(A)}} \cdot \liminf_{t\to\infty} \frac{\norm[\big]{w(A) e^{At}}}{t^{\overline{l}} 
    e^{\overline{\lambda} t}} \geq \frac{\norm{M_{1,\overline{l}}} \cdot \abs[\big]{w(\lambda_1)}}{\norm[\big]{w(A)}}
    >0. \qedhere
  \end{equation*}
\end{proof}

































    \section{Twierdzenie o minimach funkcji Lapunowa i~stabilności}
      %!TEX root = ../RRZ.tex
%
\begin{definition}
  Niech $y' = f(y)$, $f \in C^1(U)$. Funkcję $G \colon U \to \R$ nazywamy \emph{całką pierwszą}, jeśli dla każdego 
  rozwiązania $y(t)$, złożenie $G \p[\big]{y(t)}$ jest stałe.
\end{definition}
%
\begin{lemma}
  $G$ jest całką pierwszą wtedy i tylko wtedy, gdy $\langle \nabla G, f \rangle \equiv 0$.
\end{lemma}
%
\begin{definition}
	\emph{Funkcją Lapunowa} nazywamy funkcję $L$ taką, że w warunku całki powyżej zastępujemy
	$L(y(t)) = const$ przez $L(y(t))$ nierosnące. W terminach potoków ten warunek wyrażony jest przez 
	$t_1 > t_2 \implies L(\phi^{t_1}(y_0)) \le L(\phi^{t_2}(y_0))$.
\end{definition}
%
\begin{theorem}
	%
	Niech $y' = f(y)$, gdzie $f \in C^1(U)$, zaś $L$ jest funkcją Lapunowa na $U$, przy czym $L$ ma ścisłe minimum
	globalne w $y_0 \in U$. Wtedy $y_0$ jest stabilnym (w sensie Lapunowa) położeniem równowagi.
	%
\end{theorem}
%
\begin{proof}
	%
	Pokażemy, że dla każdego $t \ge 0$ zachodzi $\phi^t(y_0) = y_0$. Jeśliby tak nie było, to dla pewnego $t > 0$
	mielibyśmy $\phi(t)(y_0) \ne y_0$. Wtedy jednak $L(\phi^t(y_0)) > L(y_0) = \phi^{t_0}(y_0)$, co przeczyłoby, że
	$L$ jest funkcją Lapunowa. Zatem $y_0$ jest położeniem równowagi.
	
	Pokażemy, że
	%
	\begin{equation*}
	%
	\forall_{\eps > 0} \, \exists_{\delta > 0} \, \forall_{t \ge 0} \, \norm{ y - y_0} \le \delta \land \phi^t(y_0) \,  istnieje  \implies \norm{ \phi^t( y_0 ) - y_0 } \le \eps .
	%
	\end{equation*}
	
	Ustalmy $\eps > 0$. Bez straty ogólności załóżmy, że $\overline{B}(y_0, \eps) \subset U$ (jeśliby tak nie było, wystarczy 
	wziąć mniejszy $\eps$) oraz $L(y_0) = 0$. Wtedy zbiór $\{ y \in U : \norm{ y - y_0 } = \eps \}$ jest zbiorem zwartym.
	Wobec tego $\mu \coloneqq \inf\{ L(y) : \norm{ y - y_0 } = \eps \}$ jest przyjmowane w pewnym punkcie $y_{\min}$ oraz
	$\mu = L(y_{\min}) > L(y_0) = 0$.
	
	Z ciągłości $L$ w 0 istnieje $\delta > 0$ taka, że jeśli $\norm{ y - y_0 } \le \delta$, to $L(y) > \mu$. Gdyby dla pewnego 
	$t > 0$ zachodziło $\norm{ y - y_0 } \ge \eps$, to z własności Darboux istnieje $\tau \colon 0 \le \tau \le t$ takie, że 
	$\norm{ \phi^\tau(y) - y_0 } = \eps$. Wtedy jednak $L(\phi^\tau(y)) \le \mu > L(y) = L(\phi^0(y))$, co jest sprzecznością
	z założeniem, ze $L$ to funkcja Lapunowa. Otrzymujemy zatem, że dla każdego $t>0$ zachodzi $\norm{ y - y_0 } < \eps$.
	
	Pozostaje pokażać, że $\phi^t(y) = y(t)$ jest określone dla każdego $t \ge 0$. Jeśli $\alpha \ge 0$, to $y([0, \alpha]) \subset \overline{B}(y_0, \eps)$. Skoro $\overline{B}(y_0, \eps)$ jest zbiorem zwartym, to $y(t)$ przedłuża się na pewne
	prawostronne otoczenie $\alpha$ z lematu o przedłużaniu prze koniec. Tym samym prawym końcem dziedziny rozwiązania
	wysoconego $y(t)$ jest $+\infty$. 
	%
\end{proof}
%
    
  \chapter{Zagadnienia}
    \section{Istnienie i jednoznaczność rozwiązań, rozwiązania wysycone (otwarte)}
      \begin{theorem}[Peano]
  Niech $y' = f(y,t)$, gdzie $y(t_0) = y_0$ oraz
  \[
    f \colon H = \overline{B} (y_0, b) \times [t_0 - a, t_0 + a] \longrightarrow \R^m.
  \]
  Załóżmy, że funkcja $f$ jest ciągła i oznaczmy
  \[
    M = \sup \big\{ \norm[\big]{f(y,t)} : (y, t) \in H \big\}.
  \]
  Wówczas dla $\alpha = \min(a, b/M)$ istnieje rozwiązanie $y(t)$ określone na
  przedziale $[t_0 - \alpha, t_0 + \alpha]$, spełniające warunek początkowy $y(t_0) = y_0$.
\end{theorem}

\begin{theorem}[Picard — Lindelöf]
  Niech $y' = f(y,t)$, gdzie $y(t_0) = y_0$ oraz
  \[
    f \colon H = \overline{B} (y_0, b) \times [t_0 - a, t_0 + a] \longrightarrow \R^m.
  \]
  Załóżmy, że funkcja $f$ jest ciągła oraz lipszycowska ze względu na $y$, to znaczy
  \[
    \exists L \ \forall (y_1,t), (y_2,t) \in H \quad
    \norm[\big]{f(y_1,t) - f(y_2,t)} \leq L \cdot \norm{y_1 - y_2}.
  \]
  Oznaczmy ponadto
  \[
    M = \sup \big\{ \norm[\big]{f(y,t)} : (y, t) \in H \big\}.
  \]
  Wówczas dla dowolnego $\alpha < \min(a, b/M, 1/L)$ istnieje dokładnie jedno rozwiązanie
  zagadnienia Cauchy'ego z warunkiem początkowym $y(t_0) = y_0$ określone na przedziale
  $[t_0 - \alpha, t_0 + \alpha]$.
\end{theorem}

\begin{proof}
  Jako ćwiczenie.
\end{proof}
    \section{Metoda Frobeniusa (metoda)}
      %!TEX root = ../RRZ.tex
%
Rozważmy regularne\footnote{$a_2(t) \not= 0$} równanie różniczkowe postaci
%
\begin{equation} \label{regrr1}
  a_2(t) y'' + a_1(t) y' + a_0(t) y = 0.
\end{equation}
%
\begin{definition}
  Punkt $t_0$ nazwiemy regularnie osobliwym, jeśli funkcja $\frac{a_2(t)}{(t-t_0)^2}$ jest analityczna w $t_0$ 
  i nie znika w $t_0$, a funkcja $\frac{a_1(t)}{t-t_0}$ jest analityczna w $t_0$.
\end{definition}
%
W przypadku punktu regularnie osobliwego równanie \eqref{regrr1} sprowadza się do
%
\begin{equation} \label{regrr2}
  (t-t_0)^2 y'' + (t-t_0) p(t) y' + q(t) y = 0.
\end{equation}
%
Rozwiązań będziemy szukali jedynie poza $t_0$, $t > t_0$.
%
\paragraph{Metoda Frobeniusa.} Niech $t_0 = 0$. Szukamy rozwiązań w postaci
%
\begin{equation*}
  y(t) = t^{\lambda} \sum_{n=0}^{\infty} c_n t^n, \quad c_0 \not= 0.
\end{equation*}
%
Różniczkując stronami, dostajemy
%
\begin{flalign*}
  t y'(t) &= t^{\lambda} \sum_{n=0}^{\infty} c_n (n+\lambda) t^n, & \\
  t^2 y''(t) &= t^{\lambda} \sum_{n=0}^{\infty} c_n (n+\lambda) (n+\lambda-1) t^n.
\end{flalign*}
%
Niech $p(t) = \sum_{n=0}^{\infty} p_n t^n$ oraz $q(t) = \sum_{n=0}^{\infty} q_n t^n$. Wtedy
%
\begin{flalign*}
  t p(t) y'(t) &= t^{\lambda} \p*{\sum_{n=0}^{\infty} p_n t^n} \cdot \p*{\sum_{n=0}^{\infty} c_n (n+\lambda) t^n}
  = t^{\lambda} \sum_{n=0}^{\infty} t^n \sum_{k=0}^n c_k (k+\lambda) p_{n-k} & \\
  &= t^{\lambda} \sum_{n=0}^{\infty} t^n \p*{c_n (n+\lambda) p_0 + \sum_{k=0}^{n-1} c_k (k+\lambda) p_{n-k}}, & \\
  q(t) y(t) &= t^{\lambda} \p*{\sum_{n=0}^{\infty} q_n t^n} \cdot \p*{\sum_{n=0}^{\infty} c_n t^n}
  = t^{\lambda} \sum_{n=0}^{\infty} t^n \sum_{k=0}^n c_k q_{n-k} & \\
  &= t^{\lambda} \sum_{n=0}^{\infty} t^n \p*{c_n q_0 + \sum_{k=0}^{n-1} c_k q_{n-k}}. &
\end{flalign*}
%
Wstawiamy wynik do równania \eqref{regrr2}, otrzymując
%
\begin{multline*}
  t^{\lambda} \sum_{n=0}^{\infty} t^n \p[\big]{(n+\lambda) (n+\lambda-1) c_n + (n+\lambda) p_0 c_n + q_0 c_n } + \\
  + t^{\lambda} \sum_{n=0}^{\infty} t^n \underbrace{\p*{\sum_{k=0}^{n-1} (k+\lambda) c_k p_{n-k} + 
    \sum_{k=0}^{n-1} c_k q_{n-k}}}_{-X_n(c_0,\ldots,c_{n-1},\lambda)} = 0.
\end{multline*}
%
Dla każdej naturalnej liczby $n > 0$ zachodzi:
\begin{equation*}
  c_n \p[\big]{(n+\lambda) (n+\lambda-1) + (n+\lambda) p_0 + q_0} = X_n(c_0,\ldots,c_{n-1},\lambda)
\end{equation*}
%
oraz $X_0 = 0$ dla $n = 0$. Zdefiniujmy wielomian indeksowy $P$ wzorem
%
\begin{equation*}
  P(s) = s(s-1) + p_0 s + q_0.
\end{equation*}
%
Wtedy otrzymujemy
%
\begin{equation*}
  c_n P(n+\lambda) = X_n(c_0,\ldots,c_{n-1},\lambda) \implies c_0 P(\lambda) = 0.
\end{equation*}
%
Założyliśmy, że $c_0 \not= 0$, więc \emph{$\lambda$ musi być pierwiastkiem wielomianu indeksowego}. Dla $n = 0$ mamy 
$X_0 = 0$, więc dla $n > 0$ rekurencja przyjmuje postać
%
\begin{equation*}
  c_n = \frac{X_n(c_0,\ldots,c_{n-1},\lambda)}{P(n+\lambda)}.
\end{equation*}
%
To pozwala wyliczyć współczynniki $c_n$ dla $n > 0$, chyba że $P(n+\lambda) = 0$ dla pewnego $n$, czyli $n+\lambda$ 
jest pierwiastkiem wielomianu indeksowego.
%
\paragraph{Przypadek podstawowy.} Wielomian indeksowy ma dwa pierwiastki rzeczywiste nieróżniące się o liczbę 
całkowitą. Wówczas otrzymujemy dwa liniowo niezależne rozwiązanie przyjmując $\lambda = \lambda_1, \lambda_2$.
%
\paragraph{Przypadek zespolony.} Bierzemy jeden z nich, dostając rozwiązanie zespolone
%
\begin{equation*}
  \widetilde{y}(t) = t^{\lambda} \sum_{n=0}^{\infty} c_n t^n = e^{at} (\cos bt + i \sin bt) \sum_{n=0}^{\infty} c_n t^n.
\end{equation*}
%
Rozwiązaniami są $\real \widetilde{y}(t)$, $\imag \widetilde{y}(t)$, i są liniowo niezależne.
%
\paragraph{Przypadek pierwiastków postaci $\lambda$, $\lambda+r$, gdzie $\lambda \in \R$, $r \in \N$.} W tym przypadku 
otrzymujemy jedno rozwiązanie postaci
%
\begin{equation} \label{yzero}
  y_0(t) = t^{\lambda+r} \sum_{n=0}^{\infty} c_n t^n.
\end{equation}
%
Drugiego rozwiązania szukamy w postaci
%
\begin{equation*}
  y(t) = t^{\lambda} \sum_{n=0}^{\infty} d_n t^n + \gamma y_0(t) \ln t,
\end{equation*}
%
gdzie $\gamma$ to stała, którą wyznaczymy. Wstawiając do równania, otrzymujemy:
%
\begin{equation*}
  t^{\lambda} \sum_{n=0}^{\infty} t^n \p[\big]{P(n+\lambda) \cdot d_n - X_n(d_0,\ldots,d_{n-1},\lambda)} + \Gamma(t) =0,
\end{equation*}
%
gdzie
%
\begin{align*}
  \Gamma(t) &= \gamma \p[\big]{t^2\p[\big]{y_0(t) \ln t}'' + t p(t) \p[\big]{y_0(t) \ln t}'' +q(t) y_0(t) \ln t} \\
  &= \gamma \ln t \p[\big]{t^2 y_0''(t) + t p(t) y_0'(t) + q(t) y_0(t)}
    + \gamma \p[\big]{2t y_0'(t) + \p[\big]{p(t) - 1} y_0(t)}
\intertext{Pierwszy człon się zeruje, bo $y_0$ jest rozwiązaniem. Wstawiając \eqref{yzero}, mamy}
  &= \gamma \p*{2t \cdot \sum_{n=0}^{\infty} (\lambda+n+r) c_n t^{n+\lambda+r-1} + 
    \p[\big]{p(t)-1} \cdot \sum_{n=0}^{\infty} c_n t^{\lambda+n+r}} \\
  &= \gamma t^{\lambda+r} \p*{\sum_{n=0}^{\infty} 2(\lambda+r+n) c_n t^n + p(t) \sum_{n=0}^{\infty} c_n t^n - 
  \sum_{n=0}^{\infty} c_n t^n} \\
  &= \gamma t^{\lambda+r} \p*{\sum_{n=0}^{\infty} 2(\lambda+r+n) c_n t^n + 
    \sum_{n=0}^{\infty} t^n \sum_{k=0}^n c_k p_{n-k} - \sum_{n=0}^{\infty} c_n t^n} \\
  &= \gamma t^{\lambda+r} \sum_{n=0}^{\infty} t^n \p*{2(\lambda+r+n) c_n + \sum_{k=0}^n c_k p_{n-k} - c_n} \\
  &= \gamma t^{\lambda+r} \sum_{n=0}^{\infty} t^n \p*{2(\lambda+r+n) c_n + 
    (p_0-1) c_n + \sum_{k=0}^{n-1} c_k p_{n-k}} \\
\intertext{Zauważmy, że $P'(\lambda+n+r) = 2s + p_0 - 1$. Stąd}  
  &= \gamma t^{\lambda+r} \sum_{n=0}^{\infty} t^n \p[\big]{P'(\lambda+r+n) c_n + Y_n(c_0,\ldots,c_{n-1})}
\intertext{gdzie $Y_n(c_0,\ldots,c_{n-1}) = \sum_{k=0}^{n-1} c_k p_{n-k}$. Przesuwając indeksy, dostajemy}
  &= \gamma t^{\lambda} \sum_{n=r}^{\infty} t^n \p[\big]{Y_{n-r}(c_0,\ldots,c_{n-r-1}) + c_{n-r} P'(\lambda+n)}.
\end{align*}
%
Otrzymaliśmy zatem
%
\begin{multline*}
  \sum_{n=0}^{\infty} t^{\lambda+n} P(n+\lambda) d_n - X_n(d_0,\ldots,d_{n-1}) + \\ +
  \gamma \sum_{n=r}^{\infty} t^{\lambda+n} P'(n+\lambda) c_{n-r} + Y_{n-r}(c_0,\ldots,c_{n-r}) = 0.
\end{multline*}
%
Przyrównujmy do zera współczynniki przy $t^{\lambda+n}$. Jeśli $0 \leq n < r$, to
%
\begin{equation*}
  P(n+\lambda) d_n = X_n(d_0,\ldots,d_{n-1}).
\end{equation*}
%
Z kolei jeśli $n \geq r$, to mamy
%
\begin{equation*}
  P(n+\lambda) d_n + \gamma P'(\lambda+n) c_{n-r} = X_n(d_0,\ldots,d_{n-1}) + Y_{n-r}(c_0,\ldots,c_{n-r}).
\end{equation*}
%
%\textsc{Przypadek 1.} Dla $r>0$ dostajemy
%
%\begin{equation*}
%  
%\end{equation*}

































    \section{Rozwiązania układów liniowych jednorodnych (otwarte)}
      %!TEX root = ../RRZ.tex
%
\begin{definition}
  Równaniem liniowym nazywamy równanie postaci
  %
  \begin{equation*}
    \frac{dy}{dt} = A(t)y + B(t),
  \end{equation*}
  %
  gdzie $A(t)$ jest macierzą $m \times m$, a $B(t)$ wektorem z $\R^m$ o ciągłych współczynnikach, określonym na 
  przedziale otwartym $I \subset \R$.
\end{definition}
%
\begin{definition}
  Równanie $y' = A(t)y$ nazywamy \emph{jednorodnym}, a równanie $y' = A(t)y + B(t)$ (odpowiadającym) niejednorodnym.
\end{definition}

\begin{theorem}
  Zbiór rozwiązań równania jednorodnego jest podprzestrzenią liniową $C^0(I,\R^m)$, a zbiór rozwiązań równania 
  niejednorodnego jej warstwą.
\end{theorem}

\noindent Niech $V$ oznacza zbiór rozwiązań wysyconych równania $y' = A(t)y$.

%\begin{statement}
%  Następujące warunki są równoważne dla zbioru $\{y_1, \ldots, y_n\} \subset V$:
%  \begin{enumerate}
%    \item Zbiór jest liniowo niezależny.
%    \item Dla dowolnego $t \in I$ zbiór $\big\{y_1(t), \ldots, y_n(t)\big\}$ jest liniowo niezależny w $\R^m$.
%    \item Istnieje $t \in I$, że zbiór $\big\{y_1(t), \ldots, y_n(t)\big\}$ jest liniowo niezależny w $\R^m$.
%  \end{enumerate}
%\end{statement}
    \section{Rozwiązywanie równań liniowych niejednorodnych}
      %!TEX root = ../RRZ.tex
    \section{Hiperboliczność i stabilność punktów równowagi}
      %!TEX root = ../RRZ.tex
%

    \section{Zagadnienia brzegowe}
      \input{subfiles/2-6.tex}
  
  \chapter{Przykłady}
    \section{Rozwiązywanie równań metodą szeregów potęgowych}
    \section{Równania na wariację}
    \section{Potoki - policzenie i zastosowanie własności w konkretnych sytuacjach}
    \section{Wzory Liouville'a i Abela}
    \section{Zastosowania twierdzenia spektralnego, macierze spektralne}
    \section{Całki pierwsze, funkcje Lapunowa - zastosowanie do badania stabilności}
\end{document}
































